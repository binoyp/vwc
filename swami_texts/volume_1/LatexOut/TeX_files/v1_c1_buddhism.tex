\section{BUDDHISM, THE FULFILMENT OF HINDUISM}
\begin{center}\textit{ \textit{26th September, 1893}}\end{center}

I am not a Buddhist, as you have heard, and yet I am. If
China, or Japan, or Ceylon follow the teachings of the Great Master,
India worships him as God incarnate on earth. You have just now heard
that I am going to criticise Buddhism, but by that I wish you to
understand only this. Far be it from me to criticise him whom I worship
as God incarnate on earth. But our views about Buddha are that he was
not understood properly by his disciples. The relation between Hinduism
(by Hinduism, I mean the religion of the Vedas) and what is called
Buddhism at the present day is nearly the same as between Judaism and
Christianity. Jesus Christ was a Jew, and Shâkya Muni was a Hindu. The
Jews rejected Jesus Christ, nay, crucified him, and the Hindus have
accepted Shâkya Muni as God and worship him. But the real difference
that we Hindus want to show between modern Buddhism and what we should
understand as the teachings of Lord Buddha lies principally in this:
Shâkya Muni came to preach nothing new. He also, like Jesus, came to
fulfil and not to destroy. Only, in the case of Jesus, it was the old
people, the Jews, who did not understand him, while in the case of
Buddha, it was his own followers who did not realise the import of his
teachings. As the Jew did not understand the fulfilment of the Old
Testament, so the Buddhist did not understand the fulfilment of the
truths of the Hindu religion. Again, I repeat, Shâkya Muni came not to
destroy, but he was the fulfilment, the logical conclusion, the logical
development of the religion of the Hindus.\\

The religion of the Hindus is divided into two parts: the
ceremonial and the spiritual. The spiritual portion is specially
studied by the monks.\\

In that there is no caste. A man from the highest caste and a
man from the lowest may become a monk in India, and the two castes
become equal. In religion there is no caste; caste is simply a social
institution. Shâkya Muni himself was a monk, and it was his glory that
he had the large-heartedness to bring out the truths from the hidden
Vedas and through them broadcast all over the world. He was the first
being in the world who brought missionarising into practice — nay, he
was the first to conceive the idea of proselytising.\\

The great glory of the Master lay in his wonderful sympathy
for everybody, especially for the ignorant and the poor. Some of his
disciples were Brahmins. When Buddha was teaching, Sanskrit was no more
the spoken language in India. It was then only in the books of the
learned. Some of Buddha's Brahmins disciples wanted to translate his
teachings into Sanskrit, but he distinctly told them, "I am for the
poor, for the people; let me speak in the tongue of the people." And so
to this day the great bulk of his teachings are in the vernacular of
that day in India.\\

Whatever may be the position of philosophy, whatever may be
the position of metaphysics, so long as there is such a thing as death
in the world, so long as there is such a thing as weakness in the human
heart, so long as there is a cry going out of the heart of man in his
very weakness, there shall be a faith in God.\\

On the philosophic side the disciples of the Great Master
dashed themselves against the eternal rocks of the Vedas and could not
crush them, and on the other side they took away from the nation that
eternal God to which every one, man or woman, clings so fondly. And the
result was that Buddhism had to die a natural death in India. At the
present day there is not one who calls oneself a Buddhist in India, the
land of its birth.\\

But at the same time, Brahminism lost something — that
reforming zeal, that wonderful sympathy and charity for everybody, that
wonderful heaven which Buddhism had brought to the masses and which had
rendered Indian society so great that a Greek historian who wrote about
India of that time was led to say that no Hindu was known to tell an
untruth and no Hindu woman was known to be unchaste.\\

Hinduism cannot live without Buddhism, nor Buddhism without
Hinduism. Then realise what the separation has shown to us, that the
Buddhists cannot stand without the brain and philosophy of the
Brahmins, nor the Brahmin without the heart of the Buddhist. This
separation between the Buddhists and the Brahmins is the cause of the
downfall of India. That is why India is populated by three hundred
millions of beggars, and that is why India has been the slave of
conquerors for the last thousand years. Let us then join the wonderful
intellect of the Brahmins with the heart, the noble soul, the wonderful
humanising power of the Great Master.\\

