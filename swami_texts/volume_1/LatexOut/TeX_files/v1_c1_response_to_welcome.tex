\section{RESPONSE TO WELCOME}
\begin{center}\textit{ \textit{At the World's
Parliament of
Religions, Chicago }
 \textit{11th September, 1893}}\end{center}

Sisters and Brothers of America,\\

It fills my heart with joy unspeakable to rise in
response to
the warm and cordial welcome which you have given us. I thank you in
the name of the most ancient order of monks in the world; I thank you
in the name of the mother of religions; and I thank you in the name of
millions and millions of Hindu people of all classes and sects.\\

My thanks, also, to some of the speakers on this
platform who,
referring to the delegates from the Orient, have told you that these
men from far-off nations may well claim the honour of bearing to
different lands the idea of toleration. I am proud to belong to a
religion which has taught the world both tolerance and universal
acceptance. We believe not only in universal toleration, but we accept
all religions as true. I am proud to belong to a nation which has
sheltered the persecuted and the refugees of all religions and all
nations of the earth. I am proud to tell you that we have gathered in
our bosom the purest remnant of the Israelites, who came to Southern
India and took refuge with us in the very year in which their holy
temple was shattered to pieces by Roman tyranny. I am proud to belong
to the religion which has sheltered and is still fostering the remnant
of the grand Zoroastrian nation. I
will quote to you, brethren, a few lines from a hymn which I remember
to have repeated from my earliest boyhood, which is every day repeated
by millions of human beings: “ \textit{As the different streams having
their sources in different places all mingle their water in the sea,
so, O Lord, the different paths which men take through different
tendencies, various though they appear, crooked or straight, all lead
to Thee}.”\\

The present convention, which is one of the most
august
assemblies ever held, is in itself a vindication, a declaration to the
world of the wonderful doctrine preached in the Gita: “ \textit{Whosoever
comes to Me, through whatsoever form, I reach him; all men are
struggling through paths which in the end lead to me}.”
Sectarianism, bigotry, and its horrible descendant, fanaticism, have
long possessed this beautiful earth. They have filled the earth with
violence, drenched it often and often with human blood, destroyed
civilisation and sent whole nations to despair. Had it not been for
these horrible demons, human society would be far more advanced than it
is now. But their time is come; and I fervently hope that the bell that
tolled this morning in honour of this convention may be the death-knell
of all fanaticism, of all persecutions with the sword or with the pen,
and of all uncharitable feelings between persons wending their way to
the same goal.\\

