\section{ADDRESS AT THE FINAL SESSION}
\begin{center}\textit{ \textit{27th September, 1893}}\end{center}

The World's Parliament of Religions has become an accomplished
fact, and the merciful Father has helped those who laboured to bring it
into existence, and crowned with success their most unselfish labour.\\

My thanks to those noble souls whose large hearts and love of
truth first dreamed this wonderful dream and then realised it. My
thanks to the shower of liberal sentiments that has overflowed this
platform. My thanks to his enlightened audience for their uniform
kindness to me and for their appreciation of every thought that tends
to smooth the friction of religions. A few jarring notes were heard
from time to time in this harmony. My special
thanks to them, for they have, by
their striking contrast, made general harmony the sweeter.\\

Much has been said of the common ground of religious unity. I
am not going just now to venture my own theory. But if any one here
hopes that this unity will come by the triumph of any one of the
religions and the destruction of the others, to him I say, “Brother,
yours is an impossible hope.” Do I wish that the Christian would become
Hindu? God forbid. Do I wish that the Hindu or Buddhist would become
Christian? God forbid.\\

The seed is put in the ground, and earth and air and water are
placed around it. Does the seed become the earth; or the air, or the
water? No. It becomes a plant, it develops after the law of its own
growth, assimilates the air, the earth, and the water, converts them
into plant substance, and grows into a plant.\\

Similar is the case with religion. The Christian is not to
become a Hindu or a Buddhist, nor a Hindu or a Buddhist to become a
Christian. But each must assimilate the spirit of the others and yet
preserve his individuality and grow according to his own law of growth.\\

If the Parliament of Religions has shown anything to the world
it is this: It has proved to the world that holiness, purity and
charity are not the exclusive possessions of any church in the world,
and that every system has produced men and women of the most exalted
character. In the face of this evidence, if anybody dreams of the
exclusive survival of his own religion and the destruction of the
others, I pity him from the bottom of my heart, and point out to him
that upon the banner of every religion will soon be written, in spite
of resistance: "Help and not Fight," "Assimilation and not
Destruction," "Harmony and Peace and not Dissension."\\

