\section{PAPER ON HINDUISM}
\begin{center}\textit{Read at the Parliament on 19th
September, 1893}\end{center}

Three religions now stand in the world which have come down to
us from time prehistoric — Hinduism,  Zoroastrianism and
Judaism. They have all received tremendous shocks and all of them prove
by their survival their internal strength. But while Judaism failed to
absorb Christianity and was driven out of its place of birth by its
all-conquering daughter, and a handful of Parsees is all that remains
to tell the tale of their grand religion, sect after sect arose in
India and seemed to shake the religion of the Vedas to its very
foundations, but like the waters of the seashore in a tremendous
earthquake it receded only for a while, only to return in an
all-absorbing flood, a thousand times more vigorous, and when the
tumult of the rush was over, these sects were all sucked in, absorbed,
and assimilated into the immense body of the mother faith.\\

From the high spiritual flights of the Vedanta philosophy, of
which the latest discoveries of science seem like echoes, to the low
ideas of idolatry with its multifarious mythology, the agnosticism of
the Buddhists, and the atheism of the Jains, each and all have a place
in the Hindu's religion.\\

Where then, the question arises, where is the common centre to
which all these widely diverging radii converge? Where is the common
basis upon which all these seemingly hopeless contradictions rest? And
this is the question I shall attempt to answer.
\\

The Hindus have received their religion through revelation,
the Vedas. They hold that the Vedas are without beginning and without
end. It may sound ludicrous to this audience, how a book can be without
beginning or end. But by the Vedas no books are meant. They mean the
accumulated treasury of spiritual laws discovered
by different persons in different
times. Just as the law of gravitation existed before
its discovery, and would exist if all humanity forgot it, so is it with
the laws that
govern the spiritual world. The moral, ethical, and spiritual relations
between soul
and soul and between individual spirits and the Father of all spirits,
were there
before their discovery, and would remain even if we forgot them.\\

The discoverers of these laws are called Rishis, and we honour
them as perfected beings. I am glad to tell this audience that some of
the very greatest of them were women. Here it may be said that these
laws as laws may be without end, but they must have had a beginning.
The Vedas teach us that creation is without beginning or end. Science
is said to have proved that the sum total of cosmic energy is always
the same. Then, if there was a time when nothing existed, where was all
this manifested energy? Some say it was in a potential form in God. In
that case God is sometimes potential and sometimes kinetic, which would
make Him mutable. Everything mutable is a compound, and everything
compound must undergo that change which is called destruction. So God
would die, which is absurd. Therefore there never was a time when there
was no creation.\\

If I may be allowed to use a simile, creation and 
creator are two lines, without beginning and without end, running
parallel to each other. God is the ever active providence, by whose
power systems after systems are being evolved out of chaos, made to run
for a time and again destroyed. This is what the Brâhmin boy repeats
every day: " \textit{The sun and the moon, the Lord created like the
suns and moons of previous cycles}." And this agrees with
modern science.\\

Here I stand and if I shut my eyes, and try to conceive my
existence, "I", "I", "I", what is the idea before me? The idea of a
body. Am I, then, nothing but a combination of material
substances? The Vedas declare,
“No”. I am a spirit living in a
body. I am not the body. The body will die,
but I shall not die. Here am I in this body; it will fall, but I shall
go on living.
I had also a past. The soul was not created, for creation means a
combination which
means a certain future dissolution. If then the soul was created, it
must die. Some
are born happy, enjoy perfect health, with beautiful body, mental
vigour and all wants
supplied. Others are born miserable, some are without hands or feet,
others again are
idiots and only drag on a wretched existence. Why, if they are all
created, why does a
just and merciful God create one happy and another unhappy, why is He
so partial? Nor
would it mend matters in the least to hold that those who are miserable
in this life
will be happy in a future one. Why should a man be miserable even here
in the reign of
a just and merciful God?\\

In the second place, the idea of a creator God does not
explain the anomaly, but simply expresses the cruel fiat of an
all-powerful being. There must have been causes, then, before his
birth, to make a man miserable or happy and those were his past actions.\\

Are not all the tendencies of the mind and the body accounted
for by inherited aptitude? Here are two parallel lines of existence —
one of the mind, the other of matter. If matter and its transformations
answer for all that we have, there is no necessity for supposing the
existence of a soul. But it cannot be proved that thought has been
evolved out of matter, and if a philosophical monism is inevitable,
spiritual monism is certainly logical and no less desirable than a
materialistic monism; but neither of these is necessary here.\\

There is another suggestion. Taking all these for granted, how
is it that I do not remember anything of my past life ? This can be
easily explained. I am now speaking English. It is not my mother
tongue, in fact no words of my mother tongue are now present in my
consciousness; but let me try to bring them up, and they rush in. That
shows that consciousness is only the surface of the mental ocean, and
within its depths are stored up all our experiences. Try and struggle,
they would come up and you would be conscious even of your past life.\\

This is direct and demonstrative evidence. Verification is the
perfect proof of a theory, and here is the challenge thrown to the
world by the Rishis. We have discovered the secret by which the very
depths of the ocean of memory can be stirred up — try it and you would
get a complete reminiscence of your past life.\\

So then the Hindu believes that he is a spirit. Him the sword
cannot pierce — him the fire cannot burn — him the water cannot melt —
him the air cannot dry. The Hindu believes that every soul is a circle
whose circumference is nowhere, but whose centre is located in the
body, and that death means the change of this centre from body to body.
Nor is the soul bound by the conditions of matter. In its very essence
it is free, unbounded, holy, pure, and perfect. But somehow or
other it finds itself tied down to matter, and thinks of itself as
matter.\\

Why should the free, perfect, and pure being be thus under the
thraldom of matter, is the next question. How
can the perfect soul be deluded
into the belief that it is imperfect? We have been told
that the Hindus shirk the question and say that no such question can be
there.
Some thinkers want to answer it by positing one or more quasi-perfect
beings, and use
big scientific names to fill up the gap. But naming is not explaining.
The question
remains the same. How can the perfect become the quasi-perfect; how can
the pure,
the absolute, change even a microscopic particle of its nature? But the
Hindu is
sincere. He does not want to take shelter under sophistry. He is brave
enough to face
the question in a manly fashion; and his answer is: “I do not know. I
do not know
how the perfect being, the soul, came to think of itself as imperfect,
as joined to and
conditioned by matter." But the fact is a fact for all that. It is a
fact in
everybody's consciousness that one thinks of oneself as the body. The
Hindu does not
attempt to explain why one thinks one is the body. The answer that it
is the will of
God is no explanation. This is nothing more than what the Hindu says,
"I do not
know."\\

Well, then, the human soul is eternal and immortal, perfect
and infinite, and death means only a change of centre from one body to
another. The present is determined by our past actions, and the future
by the present. The soul will go on evolving up or reverting back from
birth to birth and death to death. But here is another question: Is man
a tiny boat in a tempest, raised one moment on the foamy crest of a
billow and dashed down into a yawning chasm the next, rolling to and
fro at the mercy of good and bad actions — a powerless, helpless wreck
in an ever-raging, ever-rushing, uncompromising current of cause and
effect; a little moth placed under the wheel of causation which rolls
on crushing everything in its way and waits not for the widow's tears
or the orphan's cry? The heart sinks at the idea, yet this is the law
of Nature. Is there no hope? Is there no escape? — was the
cry that went up from the bottom
of the heart of despair. It reached the throne of
mercy, and words of hope and consolation came down and inspired a Vedic
sage, and he
stood up before the world and in trumpet voice proclaimed the glad
tidings: "Hear,
ye children of immortal bliss! even ye that reside in higher spheres! I
have found the
Ancient One who is beyond all darkness, all delusion: knowing Him alone
you shall be
saved from death over again." "Children of immortal bliss" — what
a sweet, what a hopeful name! Allow me to call you, brethren, by that
sweet name —
heirs of immortal bliss — yea, the Hindu refuses to call you sinners.
Ye are the
Children of God, the sharers of immortal bliss, holy and perfect
beings. Ye divinities
on earth — sinners! It is a sin to call a man so; it is a standing
libel on human
nature. Come up, O lions, and shake off the delusion that you are
sheep; you are souls
immortal, spirits free, blest and eternal; ye are not matter, ye are
not bodies; matter
is your servant, not you the servant of matter. \\

Thus it is that the Vedas proclaim not a dreadful combination
of unforgiving laws, not an endless prison of cause and effect, but
that at the head of all these laws, in and through every particle of
matter and force, stands One "by whose command the wind blows, the fire
burns, the clouds rain, and death stalks upon the earth."\\

He is everywhere, the pure and formless One, the Almighty and
the All-merciful. "Thou art our father, Thou art our mother, Thou art
our beloved friend, Thou art the source of all strength; give us
strength. Thou art He that beareth the burdens of the universe; help me
bear the little burden of this life." Thus sang the Rishis of the
Vedas. And how to worship Him? Through love. "He is to be worshipped as
the one beloved, dearer than everything in this and the next life."\\

This is the doctrine of love declared in the Vedas, and let us
see how it is fully
developed and taught by Krishna, whom the Hindus believe to have been
God incarnate on
earth.\\

He taught that a man ought to live in this world like a lotus
leaf, which grows in
water but is never moistened by water; so a man ought to live in the
world — his
heart to God and his hands to work.\\

It is good to love God for hope of reward in this or the next
world, but it is
better to love God for love's sake, and the prayer goes: "Lord, I do
not want
wealth, nor children, nor learning. If it be Thy will, I shall go from
birth to birth,
but grant me this, that I may love Thee without the hope of reward —
love
unselfishly for love's sake." One of the disciples of Krishna, the then
Emperor
of India, was driven from his kingdom by his enemies and had to take
shelter with his
queen in a forest in the Himalayas, and there one day the queen asked
him how it was
that he, the most virtuous of men, should suffer so much misery.
Yudhishthira
answered, "Behold, my queen, the Himalayas, how grand and beautiful
they are; I
love them. They do not give me anything, but my nature is to love the
grand, the
beautiful, therefore I love them. Similarly, I love the Lord. He is the
source of
all beauty, of all sublimity. He is the only object to be loved; my
nature is to love
Him, and therefore I love. I do not pray for anything; I do not ask for
anything. Let
Him place me wherever He likes. I must love Him for love's sake. I
cannot trade in
love."\\

The Vedas teach that the soul is divine, only held in the
bondage of matter;
perfection will be reached when this bond will burst, and the word they
use for it is
therefore, Mukti — freedom, freedom from the bonds of imperfection,
freedom from
death and misery.\\

And this bondage can only fall off through the mercy of God,
and this mercy comes on
the pure. So purity is the condition of His mercy. How does that mercy
act? He reveals
Himself to the pure heart; the pure and the stainless see God, yea,
even in this life;
then and then only all the crookedness of the heart is made straight.
Then all doubt
ceases. He is no more the freak of a terrible law of causation. This is
the very
centre, the very vital conception of Hinduism. The Hindu does not want
to live upon
words and theories. If there are existences beyond the ordinary
sensuous existence, he
wants to come face to face with them. If there is a soul in him which
is not matter,
if there is an all-merciful universal Soul, he will go to Him direct.
He must see Him,
and that alone can destroy all doubts. So the best proof a Hindu sage
gives about the
soul, about God, is: "I have seen the soul; I have seen God." And that
is
the only condition of perfection. The Hindu religion does not consist
in struggles and
attempts to believe a certain doctrine or dogma, but in realising — not
in
believing, but in being and becomin.\\

Thus the whole object of their system is by constant struggle
to become perfect, to become divine, to reach God and see God, and this
reaching God, seeing God, becoming perfect even as the Father in Heaven
is perfect, constitutes the religion of the Hindus.\\

And what becomes of a man when he attains perfection? He lives
a life of bliss
infinite. He enjoys infinite and perfect bliss, having obtained the
only thing in
which man ought to have pleasure, namely God, and enjoys the bliss with
God.
So far all the Hindus are agreed. This is the common religion of all
the sects of India; but, then, perfection is absolute, and the absolute
cannot be two or three. It cannot have any qualities. It cannot be an
individual. And so when a soul becomes perfect and absolute, it
must become one with Brahman, and
it would only realise the Lord as the perfection,
the reality, of its own nature and existence, the existence absolute,
knowledge
absolute, and bliss absolute. We have often and often read this called
the losing
of individuality and becoming a stock or a stone.\\

 “He jests at scars that never felt a wound.”\\

I tell you it is nothing of the kind. If it is happiness to
enjoy the consciousness
of this small body, it must be greater happiness to enjoy the
consciousness of two
bodies, the measure of happiness increasing with the consciousness of
an increasing
number of bodies, the aim, the ultimate of happiness being reached when
it would
become a universal consciousness.\\

Therefore, to gain this infinite universal individuality, this
miserable little
prison-individuality must go. Then alone can death cease when I am
alone with life,
then alone can misery cease when I am one with happiness itself, then
alone can all
errors cease when I am one with knowledge itself; and this is the
necessary
scientific conclusion. Science has proved to me that physical
individuality is a
delusion, that really my body is one little continuously changing body
in an
unbroken ocean of matter; and Advaita (unity) is the necessary
conclusion with my
other counterpart, soul.\\

Science is nothing but the finding of unity. As soon as
science would reach perfect
unity, it would stop from further progress, because it would reach the
goal. Thus
Chemistry could not progress farther when it would discover one element
out of which
all other could be made. Physics would stop when it would be able to
fulfill its
services in discovering one energy of which all others are but
manifestations, and
the science of religion become perfect when it would discover Him who
is the one
life in a universe of death, Him who is the constant basis of an
ever-changing world.
One who
is the only Soul of which all
souls are but delusive manifestations. Thus is it,
through multiplicity and duality, that the ultimate unity is reached.
Religion can go
no farther. This is the goal of all science.\\

All science is bound to come to this conclusion in the long
run. Manifestation, and
not creation, is the word of science today, and the Hindu is only glad
that what he
has been cherishing in his bosom for ages is going to be taught in more
forcible
language, and with further light from the latest conclusions of science.\\

Descend we now from the aspirations of philosophy to the
religion of the ignorant.
At the very outset, I may tell you that there is no  \textit{polytheism}
in India.
In every temple, if one stands by and listens, one will find the
worshippers applying
all the attributes of God, including omnipresence, to the images. It is
not polytheism,
nor would the name henotheism explain the situation. "The rose called
by any
other name would smell as sweet." Names are not explanations.\\

I remember, as a boy, hearing a Christian missionary preach to
a crowd in India.
Among other sweet things he was telling them was that if he gave a blow
to their idol
with his stick, what could it do? One of his hearers sharply answered,
"If I
abuse your God, what can He do?" “You would be punished,” said the
preacher, "when you die." "So my idol will punish you when you
die," retorted the Hindu.\\

The tree is known by its fruits. When I have seen amongst them
that are called
idolaters, men, the like of whom in morality and spirituality and love
I have never
seen anywhere, I stop and ask myself, "Can sin beget holiness?"
Superstition is a great enemy of man, but bigotry is worse. Why does a
Christian go to church? Why is the cross holy? Why is the face turned
toward the sky in prayer? Why are there so many images in the
Catholic Church? Why are there so
many images in the minds of Protestants when they
pray? My brethren, we can no more think about anything without a mental
image than we
can live without breathing. By the law of association, the material
image calls up the
mental idea and  \textit{vice versa}. This is why the Hindu
uses an external symbol when
he worships. He will tell you, it helps to keep his mind fixed on the
Being to whom he
prays. He knows as well as you do that the image is not God, is not
omnipresent.
After all, how much does omnipresence mean to almost the whole world?
It stands merely
as a word, a symbol. Has God superficial area? If not, when we repeat
that
word "omnipresent", we think of the extended sky or of space, that is
all.\\

As we find that somehow or other, by the laws of our mental
constitution, we have to
associate our ideas of infinity with the image of the blue sky, or of
the sea, so we
naturally connect our idea of holiness with the image of a church, a
mosque, or a
cross. The Hindus have associated the idea of holiness, purity, truth,
omnipresence,
and such other ideas with different images and forms. But with this
difference that
while some people devote their whole lives to their idol of a church
and never rise
higher, because with them religion means an intellectual assent to
certain doctrines
and doing good to their fellows, the whole religion of the Hindu is
centred in
realisation. Man is to become divine by realising the divine. Idols or
temples or
churches or books are only the supports, the helps, of his spiritual
childhood: but on
and on he must progress.\\

He must not stop anywhere. " \textit{External worship,
material worship,}"
say the scriptures, " \textit{is the lowest stage; struggling to rise
high, mental
prayer is the next stage, but the highest stage is when the Lord has
been
realised}." Mark, the same earnest man who is kneeling before
the idol tells
you, " \textit{Him the Sun cannot express,
nor the moon, nor the stars, the
lightning cannot express Him, nor what we speak of
as fire; through Him they shine}." But he does not abuse any
one's idol or
call its worship sin. He recognises in it a necessary stage of life. " \textit{The
child is father of the man}." Would it be right for an old man
to say that
childhood is a sin or youth a sin?\\

If a man can realise his divine nature with the help of an
image, would it be right
to call that a sin? Nor even when he has passed that stage, should he
call it an error.
To the Hindu, man is not travelling from error to truth, but from truth
to truth, from
lower to higher truth. To him all the religions, from the lowest
fetishism to the
highest absolutism, mean so many attempts of the human soul to grasp
and realise the
Infinite, each determined by the conditions of its birth and
association, and each of
these marks a stage of progress; and every soul is a young eagle
soaring higher and
higher, gathering more and more strength, till it reaches the Glorious
Sun.\\

Unity in variety is the plan of nature, and the Hindu has
recognised it. Every other religion lays down certain fixed dogmas, and
tries to force society to adopt them. It places before society only one
coat which must fit Jack and John and Henry, all alike. If it does not
fit John or Henry, he must go without a coat to cover his body. The
Hindus have discovered that the absolute can only be realised, or
thought of, or stated, through the relative, and the images, crosses,
and crescents are simply so many symbols — so many pegs to hang the
spiritual ideas on. It is not that this help is necessary for every
one, but those that do not need it have no right to say that it is
wrong. Nor is it compulsory in Hinduism.\\

One thing I must tell you. Idolatry in India does not mean
anything horrible. It is not the mother of harlots. On the other hand,
it is the attempt of
undeveloped minds to grasp high
spiritual truths. The Hindus have their faults,
they sometimes have their exceptions; but mark this, they are always
for punishing
their own bodies, and never for cutting the throats of their
neighbours. If the
Hindu fanatic burns himself on the pyre, he never lights the fire of
Inquisition.
And even this cannot be laid at the door of his religion any more than
the burning
of witches can be laid at the door of Christianity.\\

To the Hindu, then, the whole world of religions is only a
travelling, a coming up,
of different men and women, through various conditions and
circumstances, to the same
goal. Every religion is only evolving a God out of the material man,
and the same God
is the inspirer of all of them. Why, then, are there so many
contradictions? They are
only apparent, says the Hindu. The contradictions come from the same
truth adapting
itself to the varying circumstances of different natures.\\

It is the same light coming through glasses of different
colours. And these little
variations are necessary for purposes of adaptation. But in the heart
of everything the
same truth reigns. The Lord has declared to the Hindu in His
incarnation as
Krishna, " \textit{I am in every religion as the thread through a
string of pearls.
Wherever thou seest extraordinary holiness and extraordinary power
raising and
purifying humanity, know thou that I am there}." And what has
been the result?
I challenge the world to find, throughout the whole system of Sanskrit
philosophy,
any such expression as that the Hindu alone will be saved and not
others. Says
Vyasa, " \textit{We find perfect men even beyond the pale of our caste
and
creed}." One thing more. How, then, can the Hindu, whose whole
fabric of
thought centres in God, believe in Buddhism which is agnostic, or in
Jainism which
is atheistic?\\

The Buddhists or the Jains do not depend upon God; but the
whole force of their
religion is directed to the great central truth in every religion, to
evolve a God out
of man. They have not seen the Father, but they have seen the Son. And
he that hath
seen the Son hath seen the Father also. \\

This, brethren, is a short sketch of the religious ideas of
the Hindus. The Hindu
may have failed to carry out all his plans, but if there is ever to be
a universal
religion, it must be one which will have no location in place or time;
which will
be infinite like the God it will preach, and whose sun will shine upon
the followers
of Krishna and of Christ, on saints and sinners alike; which will not
be Brahminic or
Buddhistic, Christian or Mohammedan, but the sum total of all these,
and still have
infinite space for development; which in its catholicity will embrace
in its infinite
arms, and find a place for, every human being, from the lowest
grovelling savage not
far removed from the brute, to the highest man towering by the virtues
of his head and
heart almost above humanity, making society stand in awe of him and
doubt his human
nature. It will be a religion which will have no place for persecution
or intolerance
in its polity, which will recognise divinity in every man and woman,
and whose whole
scope, whose whole force, will be created in aiding humanity to realise
its own true,
divine nature.\\

Offer such a religion, and all the nations will follow you.
Asoka's council was a
council of the Buddhist faith. Akbar's, though more to the purpose, was
only a
parlour-meeting. It was reserved for America to proclaim to all
quarters of the globe
that the Lord is in every religion.\\

Hail, Columbia, motherland of liberty! It has been given to
thee, who never dipped
her hand in her neighbour’s blood, who never found out that the
shortest way of
becoming rich was by robbing one’s neighbours, it has been given to
thee to march
at the vanguard of civilisation with the flag of harmony.\\

