\section{CONCENTRATION: ITS SPIRITUAL USES }
\begin{center}\textit{PATANJALI'S YOGA APHORISMS}\end{center}

\begin{center}\textit{CHAPTER I}\end{center}

\begin{center}
\begin{sanskrit}
अथ योगानुशासनम् ॥१॥
\end{sanskrit}
\end{center}
1. Now concentration is explained. \\

\begin{center}
\begin{sanskrit}
योगश्चित्तवृत्तिनिरोधः ॥२॥
\end{sanskrit}
\end{center}
2. Yoga is restraining the mind-stuff (Chitta) from taking
various forms (Vrittis). \\

A good deal of explanation is necessary here. We have to
understand what Chitta is, and what the Vrittis are. I have eyes. Eyes
do not see. Take away the brain centre which is in the head, the eyes
will still be there, the retinae complete, as also the pictures of
objects on them, and yet the eyes will not see. So the eyes are only a
secondary instrument, not the organ of vision. The organ of vision is
in a nerve centre of the brain. The two eyes will not be sufficient.
Sometimes a man is asleep with his eyes open. The light is there and
the picture is there, but a third thing is necessary — the mind must be
joined to the organ. The eye is the external instrument; we need also
the brain centre and the agency of the mind. Carriages roll down a
street, and you do not hear them. Why? Because your mind has not
attached itself to the organ of hearing. First, there is the
instrument, then there is the organ, and third, the mind attached to
these two. The mind takes the impression farther in, and presents it to
the determinative faculty — Buddhi — which reacts. Along with this
reaction flashes the idea of egoism. Then this mixture of action and
reaction is presented to the
Purusha, the real Soul, who perceives an object in this mixture. The
organs (Indriyas), together with the mind (Manas), the determinative
faculty (Buddhi), and egoism (Ahamkâra), form the group called the
Antahkarana (the internal instrument). They are but various processes
in the mind-stuff, called Chitta. The waves of thought in the Chitta
are called Vrittis (literally "whirlpool") . What is thought? Thought
is a force, as is gravitation or repulsion. From the infinite
storehouse of force in nature, the instrument called Chitta takes hold
of some, absorbs it and sends it out as thought. Force is supplied to
us through food, and out of that food the body obtains the power of
motion etc. Others, the finer forces, it throws out in what we call
thought. So we see that the mind is not intelligent; yet it appears to
be intelligent. Why? Because the intelligent soul is behind it. You are
the only sentient being; mind is only the instrument through which you
catch the external world. Take this book; as a book it does not exist
outside, what exists outside is unknown and unknowable. The unknowable
furnishes the suggestion that gives a blow to the mind, and the mind
gives out the reaction in the form of a book, in the same manner as
when a stone is thrown into the water, the water is thrown against it
in the form of waves. The real universe is the occasion of the reaction
of the mind. A book form, or an elephant form, or a man form, is not
outside; all that we know is our mental reaction from the outer
suggestion. "Matter is the permanent possibility of sensations," said
John Stuart Mill. It is only the suggestion that is outside. Take an
oyster for example. You know how pearls are made. A parasite gets
inside the shell and causes irritation, and the oyster throws a sort of
enamelling round it, and this makes the pearl. The universe of
experience is our own enamel, so to say, and the real universe is the
parasite serving as nucleus. The ordinary man will never understand it,
because when he tries to do so, he throws out
an enamel, and sees only his own enamel. Now we understand what is
meant by these Vrittis. The real man is behind the mind; the mind is
the instrument his hands; it is his intelligence that is percolating
through the mind. It is only when you stand behind the mind that it
becomes intelligent. When man gives it up, it falls to pieces and is
nothing. Thus you understand what is meant by Chitta. It is the
mind-stuff, and Vrittis are the waves and ripples rising in it when
external causes impinge on it. These Vrittis are our universe. \\

The bottom of a lake we cannot see, because its surface is
covered with ripples. It is only possible for us to catch a glimpse of
the bottom, when the ripples have subsided, and the water is calm. If
the water is muddy or is agitated all the time, the bottom will not be
seen. If it is clear, and there are no waves, we shall see the bottom.
The bottom of the lake is our own true Self; the lake is the Chitta and
the waves the Vrittis. Again, the mind is in three states, one of which
is darkness, called Tamas, found in brutes and idiots; it only acts to
injure. No other idea comes into that state of mind. Then there is the
active state of mind, Rajas, whose chief motives are power and
enjoyment. "I will be powerful and rule others." Then there is the
state called Sattva, serenity, calmness, in which the waves cease, and
the water of the mind-lake becomes clear. It is not inactive, but
rather intensely active. It is the greatest manifestation of power to
be calm. It is easy to be active. Let the reins go, and the horses will
run away with you. Anyone can do that, but he who can stop the plunging
horses is the strong man. Which requires the greater strength, letting
go or restraining? The calm man is not the man who is dull. You must
not mistake Sattva for dullness or laziness. The calm man is the one
who has control over the mind waves. Activity is the manifestation of
inferior strength, calmness, of the superior. \\

The Chitta is always trying to get back to its natural pure
state, but the organs draw it out. To restrain it, to check this
outward tendency, and to start it on the return journey to the essence
of intelligence is the first step in Yoga, because only in this way can
the Chitta get into its proper course. \\

Although the Chitta is in every animal, from the lowest to the
highest, it is only in the human form that we find it as the intellect.
Until the mind-stuff can take the form of intellect it is not possible
for it to return through all these steps, and liberate the soul.
Immediate salvation is impossible for the cow or the dog, although they
have mind, because their Chitta cannot as yet take that form which we
call intellect. \\

The Chitta manifests itself in the following forms —
scattering, darkening, gathering, one-pointed, and concentrated. The
scattering form is activity. Its tendency is to manifest in the form of
pleasure or of pain. The darkening form is dullness which tends to
injury. The commentator says, the third form is natural to the Devas,
the angels, and the first and second to the demons. The gathering form
is when it struggles to centre itself. The one-pointed form is when it
tries to concentrate, and the concentrated form is what brings us to
Samâdhi. \\

\begin{center}
\begin{sanskrit}
तदा द्रष्टुः स्वरूपेऽवस्थानम् ॥३॥
\end{sanskrit}
\end{center}
3. At that time (the time of concentration) the seer (Purusha)
rests in his own (unmodified) state. \\

As soon as the waves have stopped, and the lake has become
quiet, we see its bottom. So with the mind; when it is calm, we see
what our own nature is; we do not mix ourselves but remain our own
selves. \\

\begin{center}
\begin{sanskrit}
वृत्तिसारूप्यमितरत्र ॥४॥
\end{sanskrit}
\end{center}
4. At other times (other than that of concentration) the seer
is identified with the modifications. \\

For instance, someone blames me; this produces a modification,
Vritti, in my mind, and I identify myself with it and the result is
misery. \\

\begin{center}
\begin{sanskrit}
वृत्तयः पंचतय्यः क्लिष्टा अक्लिष्टाः ॥५॥
\end{sanskrit}
\end{center}
5. There are five classes of modifications, (some) painful and
(others) not painful. \\

\begin{center}
\begin{sanskrit}
प्रमाण-विपर्यय-विकल्प-निद्रा-स्मृतयः ॥६॥
\end{sanskrit}
\end{center}
6. (These are) right knowledge, indiscrimination, verbal
delusion, sleep, and memory. \\

\begin{center}
\begin{sanskrit}
प्रत्यक्षानुमानागमाः प्रमाणानि ॥७॥
\end{sanskrit}
\end{center}
7. Direct perception, inference, and
competent evidence are
proofs. \\

When two of our perceptions do not contradict each other, we
call it proof. I hear something, and if it contradicts something
already perceived, I begin to fight it out, and do not believe it.
There are also three kinds of proof. Pratyaksha, direct perception;
whatever we see and feel, is proof, if there has been nothing to delude
the senses. I see the world; that is sufficient proof that it exists.
Secondly, Anumâna, inference; you see a sign, and from the sign you
come to the thing signified. Thirdly, Âptavâkya, the direct evidence of
the Yogis, of those who have seen the truth. We are all of us
struggling towards knowledge. But you and I have to struggle hard, and
come to knowledge through a long tedious process of reasoning, but the
Yogi, the pure one, has gone beyond all this. Before his mind, the
past, the present, and the future are alike, one book for him to read;
he does not require to go through the tedious processes for knowledge
we have to; his words are proof, because he sees knowledge in himself.
These, for instance, are the authors of the sacred scriptures;
therefore the scriptures are proof. If any such persons are living now
their words will be proof. Other
philosophers go into long discussions about Aptavakya and they say,
"What is the proof of their words?" The proof is their direct
perception. Because whatever I see is proof, and whatever you see is
proof, if it does not contradict any past knowledge. There is knowledge
beyond the senses, and whenever it does not contradict reason and past
human experience, that knowledge is proof. Any madman may come into
this room and say he sees angels around him; that would not be proof.
In the first place, it must be true knowledge, and secondly, it must
not contradict past knowledge, and thirdly, it must depend upon the
character of the man who gives it out. I hear it said that the
character of the man is not of so much importance as what he may say;
we must first hear what he says. This may be true in other things. A
man may be wicked, and yet make an astronomical discovery, but in
religion it is different, because no impure man will ever have the
power to reach the truths of religion. Therefore we have first of all
to see that the man who declares himself to be an Âpta is a perfectly
unselfish and holy person; secondly, that he has reached beyond the
senses; and thirdly, that what he says does not contradict the past
knowledge of humanity. Any new discovery of truth does not contradict
the past truth, but fits into it. And fourthly, that truth must have a
possibility of verification. If a man says, "I have seen a vision," and
tells me that I have no right to see it, I believe him not. Everyone
must have the power to see it for himself. No one who sells his
knowledge is an Apta. All these conditions must be fulfilled; you must
first see that the man is pure, and that he has no selfish motive; that
he has no thirst for gain or fame. Secondly, he must show that he is
superconscious. He must give us something that we cannot get from our
senses, and which is for the benefit of the world. Thirdly, we must see
that it does not contradict other truths; if it contradicts other
scientific truths reject it at once. Fourthly, the man
should never be singular; he should only represent what all men can
attain. The three sorts of proof are, then, direct sense-perception,
inference, and the words of an Apta. I cannot translate this word into
English. It is not the word "inspired", because inspiration is believed
to come from outside, while this knowledge comes from the man himself.
The literal meaning is "attained". \\

\begin{center}
\begin{sanskrit}
विपर्ययो मिथ्याज्ञानमतद्रूपप्रतिष्ठम् ॥८॥
\end{sanskrit}
\end{center}
8. Indiscrimination is false knowledge not established in real
nature. \\

The next class of Vrittis that arises is mistaking one thing
for another, as a piece of mother-of-pearl is taken for a piece of
silver. \\

\begin{center}
\begin{sanskrit}
शब्दज्ञानानुपाती वस्तुशून्यो विकल्पः ॥९॥
\end{sanskrit}
\end{center}
9. Verbal delusion follows from words having no
(corresponding) reality. \\

There is another class of Vrittis called Vikalpa. A word is
uttered, and we do not wait to consider its meaning; we jump to a
conclusion immediately. It is the sign of weakness of the Chitta. Now
you can understand the theory of restraint. The weaker the man, the
less he has of restraint. Examine yourselves always by that test. When
you are going to be angry or miserable, reason it out how it is that
some news that has come to you is throwing your mind into Vrittis. \\

\begin{center}
\begin{sanskrit}
अभाव-प्रत्ययालम्बना-वृत्तिर्निद्रा ॥१०॥
\end{sanskrit}
\end{center}
10. Sleep is a Vritti which embraces the feeling of voidness. \\

The next class of Vrittis is called sleep and dream. When we
awake, we know that we have been sleeping; we can only have memory of
perception. That which we do not perceive we never can have any memory
of. Every reaction is a wave in
the lake. Now, if, during sleep, the mind had no waves, it would have
no perceptions, positive or negative, and, therefore, we would not
remember them. The very reason of our remembering sleep is that during
sleep there was a certain class of waves in the mind. Memory is another
class of Vrittis which is called Smriti. \\

\begin{center}
\begin{sanskrit}
अनुभूतविषयासम्प्रमोषः स्मृतिः ॥११॥
\end{sanskrit}
\end{center}
11. Memory is when the (Vrittis of) perceived subjects do not
slip away (and through impressions come back to consciousness). \\

Memory can come from direct perception, false knowledge,
verbal delusion, and sleep. For instance, you hear a word. That word is
like a stone thrown into the lake of the Chitta; it causes a ripple,
and that ripple rouses a series of ripples; this is memory. So in
sleep. When the peculiar kind of ripple called sleep throws the Chitta
into a ripple of memory, it is called a dream. Dream is another form of
the ripple which in the waking state is called memory.\\

\begin{center}
\begin{sanskrit}
अभ्यासवैराग्याभ्यां तन्निरोधः ॥१२॥
\end{sanskrit}
\end{center}
12. Their control is by practice and nonattachment. \\

The mind, to have non-attachment, must be clear, good, and
rational. Why should we practice? Because each action is like the
pulsations quivering over the surface of the lake. The vibration dies
out, and what is left? The Samskâras, the impressions. When a large
number of these impressions are left on the mind, they coalesce and
become a habit. It is said, "Habit is second nature", it is first
nature also, and the whole nature of man; everything that we are is the
result of habit. That gives us consolation, because, if it is only
habit, we can make and unmake it at any time. The Samskaras are left by
these vibrations passing out of
our mind, each one of them leaving its result. Our character is the
sum-total of these marks, and according as some particular wave
prevails one takes that tone. If good prevails, one becomes good; if
wickedness, one becomes wicked; if joyfulness, one becomes happy. The
only remedy for bad habits is counter habits; all the bad habits that
have left their impressions are to be controlled by good habits. Go on
doing good, thinking holy thoughts continuously; that is the only way
to suppress base impressions. Never say any man is hopeless, because he
only represents a character, a bundle of habits, which can be checked
by new and better ones. Character is repeated habits, and repeated
habits alone can reform character. \\

\begin{center}
\begin{sanskrit}
तत्र स्थितौ यत्नोऽभ्यासः ॥१३॥
\end{sanskrit}
\end{center}
13. Continuous struggle to keep them (the Vrittis) perfectly
restrained is practice. \\

What is practice? The attempt to restrain the mind in Chitta
form, to prevent its going out into waves. \\

\begin{center}
\begin{sanskrit}
स तु दीर्घकालनैरन्तर्यसत्कारासेवितो दृढभूमिः
॥१४॥
\end{sanskrit}
\end{center}
14. It becomes firmly grounded by long constant efforts with
great love (for the end to be attained). \\

Restraint does not come in one day, but by long continued
practice. \\

\begin{center}
\begin{sanskrit}
दृष्टानुश्रविकविषयवितृष्णस्य वशीकारसंज्ञा
वैराग्यम् ॥१५॥
\end{sanskrit}
\end{center}
15. That effect which comes to these who have given up their
thirst after objects, either seen or heard, and which wills to control
the objects, is non-attachment. \\

The two motive powers of our actions are (1) what we see
ourselves, (2) the experience of others. These two
forces throw the mind, the lake, into various waves. Renunciation is
the power of battling against these forces and holding the mind in
check. Their renunciation is what see want. I am passing through a
street, and a man comes and takes away my watch. That is my own
experience. I see it myself, and it immediately throws my Chitta into a
wave, taking the form of anger. Allow not that to come. If you cannot
prevent that, you are nothing; if you can, you have Vairâgya. Again,
the experience of the worldly-minded teaches us that sense-enjoyments
are the highest ideal. These are tremendous temptations. To deny them,
and not allow the mind to come to a wave form with regard to them, is
renunciation; to control the twofold motive powers arising from my own
experience and from the experience of others, and thus prevent the
Chitta from being governed by them, is Vairagya. These should be
controlled by me, and not I by them. This sort of mental strength is
called renunciation. Vairagya is the only way to freedom. \\

\begin{center}
\begin{sanskrit}
तत्परं पुरुषख्यातेर्गुणवैतृष्ण्यम् ॥१६॥
\end{sanskrit}
\end{center}
16. That is extreme non-attachment which gives up even the
qualities, and comes from the knowledge of (the real nature of) the
Purusha. \\

It is the highest manifestation of the power of Vairagya when
it takes away even our attraction towards the qualities. We have first
to understand what the Purusha, the Self, is and what the qualities
are. According to Yoga philosophy, the whole of nature consists of
three qualities or forces; one is called Tamas, another Rajas, and the
third Sattva. These three qualities manifest themselves in the physical
world as darkness or inactivity, attraction or repulsion, and
equilibrium of the two. Everything that is in nature, all
manifestations, are combinations and recombinations of these three
forces. Nature has been divided
into various categories by the Sânkhyas; the Self of man is beyond all
these, beyond nature. It is effulgent, pure, and perfect. Whatever of
intelligence we see in nature is but the reflection of this Self upon
nature. Nature itself is insentient. You must remember that the word
nature also includes the mind; mind is in nature; thought is in nature;
from thought, down to the grossest form of matter, everything is in
nature, the manifestation of nature. This nature has covered the Self
of man, and when nature takes away the covering, the self appears in
Its own glory. The non-attachment, as described in aphorism 15 (as
being control of objects or nature) is the greatest help towards
manifesting the Self. The next aphorism defines Samadhi, perfect
concentration which is the goal of the Yogi. \\

\begin{center}
\begin{sanskrit}
वितर्कविचारानन्दास्मितानुगमात् सम्प्रज्ञातः
॥१७॥
\end{sanskrit}
\end{center}
17. The concentration called right knowledge is that which is
followed by reasoning, discrimination bliss, unqualified egoism. \\

Samadhi is divided into two varieties. One is called the
Samprajnâta, and the other the Asamprajnâta. In the Samprajnata Samadhi
come all the powers of controlling nature. It is of four varieties. The
first variety is called the Savitarka, when the mind meditates upon an
object again and again, by isolating it from other objects. There are
two sorts of objects for meditation in the twenty-five categories of
the Sankhyas, (1) the twenty-four insentient categories of Nature, and
(2) the one sentient Purusha. This part of Yoga is based entirely on
Sankhya philosophy, about which I have already told you. As you will
remember, egoism and will and mind have a common basis, the Chitta or
the mind-stuff, out of which they are all manufactured. The mind-stuff
takes in the forces of nature, and projects them as thought. There must
be something, again, where both force and matter are one.
This is called Avyakta, the unmanifested state of nature before
creation, and to which, after the end of a cycle, the whole of nature
returns, to come out again after another period. Beyond that is the
Purusha, the essence of intelligence. Knowledge is power, and as soon
as we begin to know a thing, we get power over it; so also when the
mind begins to meditate on the different elements, it gains power over
them. That sort of meditation where the external gross elements are the
objects is called Savitarka. Vitarka means question; Savitarka, with
question, questioning the elements, as it were, that they may give
their truths and their powers to the man who meditates upon them. There
is no liberation in getting powers. It is a worldly search after
enjoyments, and there is no enjoyment in this life; all search for
enjoyment is vain; this is the old, old lesson which man finds so hard
to learn. When he does learn it, he gets out of the universe and
becomes free. The possession of what are called occult powers is only
intensifying the world, and in the end, intensifying suffering. Though
as a scientist Patanjali is bound to point out the possibilities of
this science, he never misses an opportunity to warn us against these
powers. \\

Again, in the very same meditation, when one struggles to take
the elements out of time and space, and think of them as they are, it
is called Nirvitarka, without question. When the meditation goes a step
higher, and takes the Tanmatras as its object, and thinks of them as in
time and space, it is called Savichâra, with discrimination; and when
in the same meditation one eliminates time and space, and thinks of the
fine elements as they are, it is called Nirvichâra, without
discrimination. The next step is when the elements are given up, both
gross and fine, and the object of meditation is the interior organ, the
thinking organ. When the thinking organ is thought of as bereft of the
qualities of activity and dullness, it is then called
Sânanda, the blissful Samadhi. When the mind itself is the object of
meditation, when meditation becomes very ripe and concentrated, when
all ideas of the gross and fine materials are given up, when the Sattva
state only of the Ego remains, but differentiated from all other
objects, it is called Sâsmita Samadhi. The man who has attained to this
has attained to what is called in the Vedas "bereft of body". He can
think of himself as without his gross body; but he will have to think
of himself as with a fine body. Those that in this state get merged in
nature without attaining the goal are called Prakritilayas, but those
who do not stop even there reach the goal, which is freedom. \\

\begin{center}
\begin{sanskrit}
विरामप्रत्ययाभ्यासपूर्वः संस्कारशेषोऽन्यः
॥१८॥
\end{sanskrit}
\end{center}
18. There is another Samadhi which is attained by the constant
practice of cessation of all mental activity, in which the Chitta
retains only the unmanifested impressions. \\

This is the perfect superconscious Asamprajnata Samadhi, the
state which gives us freedom. The first state does not give us freedom,
does not liberate the soul. A man may attain to all powers, and yet
fall again. There is no safeguard until the soul goes beyond nature. It
is very difficult to do so, although the method seems easy. The method
is to meditate on the mind itself, and whenever thought comes, to
strike it down, allowing no thought to come into the mind, thus making
it an entire vacuum. When we can really do this, that very moment we
shall attain liberation. When persons without training and preparation
try to make their minds vacant, they are likely to succeed only in
covering themselves with Tamas, the material of ignorance, which makes
the mind dull and stupid, and leads them to think that they are making
a vacuum of the mind. To be able to really do that is to
manifest the greatest strength, the highest control. When this state,
Asamprajnata, superconsciousness, is reached, the Samadhi becomes
seedless. What is meant by that? In a concentration where there is
consciousness, where the mind succeeds only in quelling the waves in
the Chitta and holding them down, the waves remain in the form of
tendencies. These tendencies (or seeds) become waves again, when the
time comes. But when you have destroyed all these tendencies, almost
destroyed the mind, then the Samadhi becomes seedless; there are no
more seeds in the mind out of which to manufacture again and again this
plant of life, this ceaseless round of birth and death. \\

You may ask, what state would that be in which there is no
mind, there is no knowledge? What we call knowledge is a lower state
than the one beyond knowledge. You must always bear in mind that the
extremes look very much alike. If a very low vibration of ether is
taken as darkness, an intermediate state as light, very high vibration
will be darkness again. Similarly, ignorance is the lowest state,
knowledge is the middle state, and beyond knowledge is the highest
state, the two extremes of which seem the same. Knowledge itself is a
manufactured something, a combination; it is not reality. \\

What is the result of constant practice of this higher
concentration? All old tendencies of restlessness and dullness will be
destroyed, as well as the tendencies of goodness too. The case is
similar to that of the chemicals used to take the dirt and alloy off
gold. When the ore is smelted down, the dross is burnt along with the
chemicals. So this constant controlling power will stop the previous
bad tendencies, and eventually, the good ones also. Those good and evil
tendencies will suppress each other, leaving alone the Soul, in its own
splendour untrammelled by either good or bad, the omnipresent,
omnipotent, and omniscient. Then the man will know that he had neither
birth nor death, nor need for
heaven or earth. He will know that he neither came nor went, it was
nature which was moving, and that movement was reflected upon the soul.
The form of the light reflected by the glass upon the wall moves, and
the wall foolishly thinks it is moving. So with all of us; it is the
Chitta constantly moving making itself into various forms, and we think
that we are these various forms. All these delusions will vanish. When
that free Soul will command — not pray or beg, but command — then
whatever It desires will be immediately fulfilled; whatever It wants It
will be able to do. According to the Sankhya philosophy, there is no
God. It says that there can be no God of this universe, because if
there were one, He must be a soul, and a soul must be either bound or
free. How can the soul that is bound by nature, or controlled by
nature, create? It is itself a slave. On the other hand, why should the
Soul that is free create and manipulate all these things? It has no
desires, so it cannot have any need to create. Secondly, it says the
theory of God is an unnecessary one; nature explains all. What is the
use of any God? But Kapila teaches that there are many souls, who,
though nearly attaining perfection, fall short because they cannot
perfectly renounce all powers. Their minds for a time merge in nature,
to re-emerge as its masters. Such gods there are. We shall all become
such gods, and, according to the Sankhyas, the God spoken of in the
Vedas really means one of these free souls. Beyond them there is not an
eternally free and blessed Creator of the universe. On the other hand,
the Yogis say, "Not so, there is a God; there is one Soul separate from
all other souls, and He is the eternal Master of all creation, the ever
free, the Teacher of all teachers." The Yogis admit that those whom the
Sankhyas call "the merged in nature" also exist. They are Yogis who
have fallen short of perfection, and though, for a time, debarred from
attaining the goal, remain as rulers of parts of the universe. \\

\begin{center}
\begin{sanskrit}
भव-प्रत्ययो विदेह-प्रकृतिलयानाम् ॥१९॥
\end{sanskrit}
\end{center}
19. (This Samadhi when not followed by extreme non-attachment)
becomes the cause of the re-manifestation of the gods and of those that
become merged in nature. \\

The gods in the Indian systems of philosophy represent certain
high offices which are filled successively by various souls. But none
of them is perfect. \\

\begin{center}
\begin{sanskrit}
श्रद्धा-वीर्य-स्मृति-समाधि-प्रज्ञा-पूर्वक
इतरेषाम् ॥२०॥
\end{sanskrit}
\end{center}
20. To others (this Samadhi) comes through faith, energy,
memory, concentration, and discrimination of the real. \\

These are they who do not want the position of gods or even
that of rulers of cycles. They attain to liberation.\\

\begin{center}
\begin{sanskrit}
तीव्रसंवेगानामासन्नः ॥२१॥
\end{sanskrit}
\end{center}
21. Success is speedy for the extremely energetic. \\

\begin{center}
\begin{sanskrit}
मृदुमध्याधिमात्रत्वात्ततोऽपि विशेषः ॥२२॥
\end{sanskrit}
\end{center}
22. The success of Yogis differs according as the means they
adopt are mild, medium, or intense. \\

\begin{center}
\begin{sanskrit}
ईश्वरप्रणिधानाद्वा ॥२३॥
\end{sanskrit}
\end{center}
23. Or by devotion to Ishvara. \\

\begin{center}
\begin{sanskrit}
क्लेशकर्मविपाकाशयैरपरामृष्टः पुरुषविशेष
ईश्वरः ॥२४॥
\end{sanskrit}
\end{center}
24. Ishvara (the Supreme Ruler) is a special Purusha,
untouched by misery, actions, their results, and desires. \\

We must again remember that the Pâtanjala Yoga philosophy is
based upon the Sankhya philosophy; only in the latter there is no place
for God, while with the Yogis God has a place. The Yogis, however, do
not mention many ideas about
God, such as creating. God as the Creator of the universe is not meant
by the Ishvara of the Yogis. According to the Vedas, Ishvara is the
Creator of the universe; because it is harmonious, it must be the
manifestation of one will. The Yogis want to establish a God, but they
arrive at Him in a peculiar fashion of their own. They say: \\

\begin{center}
\begin{sanskrit}
तत्र निरतिशयं सर्वज्ञत्वबीजम् ॥२५॥
\end{sanskrit}
\end{center}
25. In Him becomes infinite that all-knowingness which in
others is (only) a germ. \\

The mind must always travel between two extremes. You can
think of limited space, but that very idea gives you also unlimited
space. Close your eyes and think of a little space; at the same time
that you perceive the little circle, you have a circle round it of
unlimited dimensions. It is the same with time. Try to think of a
second; you will have, with the same act of perception, to think of
time which is unlimited. So with knowledge. Knowledge is only a germ in
man, but you will have to think of infinite knowledge around it, so
that the very constitution of our mind shows us that there is unlimited
knowledge, and the Yogis call that unlimited knowledge God. \\

\begin{center}
\begin{sanskrit}
स पूर्वेषामपि गुरुः कालेनानवच्छेदात् ॥२६॥
\end{sanskrit}
\end{center}
26. He is the Teacher of even the ancient teachers, being not
limited by time. \\

It is true that all knowledge is within ourselves, but this
has to be called forth by another knowledge. Although the capacity to
know is inside us, it must be called out, and that calling out of
knowledge can only be done, a Yogi maintains, through another
knowledge. Dead, insentient matter never calls out knowledge, it is the
action of knowledge that brings out knowledge. Knowing beings must be
with us to call forth what is in
us, so these teachers were always necessary. The world was never
without them, and no knowledge can come without them. God is the
Teacher of all teachers, because these teachers, however great they may
have been — gods or angels — were all bound and limited by time, while
God is not. There are two peculiar deductions of the Yogis. The first
is that in thinking of the limited, the mind must think of the
unlimited; and that if one part of that perception is true, so also
must the other be, for the reason that their value as perceptions of
the mind is equal. The very fact that man has a little knowledge shows
that God has unlimited knowledge. If I am to take one, why not the
other? Reason forces me to take both or reject both. If I believe that
there is a man with a little knowledge, I must also admit that there is
someone behind him with unlimited knowledge. The second deduction is
that no knowledge can come without a teacher. It is true, as the modern
philosophers say, that there is something in man which evolves out of
him; all knowledge is in man, but certain environments are necessary to
call it out. We cannot find any knowledge without teachers. If there
are men teachers, god teachers, or angel teachers, they are all
limited; who was the teacher before them? We are forced to admit, as a
last conclusion, one teacher who is not limited by time; and that One
Teacher of infinite knowledge, without beginning or end, is called God.
\\

\begin{center}
\begin{sanskrit}
तस्य वाचकः प्रणवः ॥२७॥
\end{sanskrit}
\end{center}
27. His manifesting word is Om. \\

None\\

\begin{center}
\begin{sanskrit}
तज्जपस्तदर्थभावनम् ॥२८॥
\end{sanskrit}
\end{center}
28. The repetition of this (Om) and meditating on its meaning
(is the way). \\

Why should there be repetition? We have not forgotten the
theory of Samskaras, that the sum-total of
impressions lives in the mind. They become more and more latent but
remain there, and as soon as they get the right stimulus, they come
out. Molecular vibration never ceases. When this universe is destroyed,
all the massive vibrations disappear; the sun, moon, stars, and earth,
melt down; but the vibrations remain in the atoms. Each atom performs
the same function as the big worlds do. So even when the vibrations of
the Chitta subside, its molecular vibrations go on, and when they get
the impulse, come out again. We can now understand what is meant by
repetition. It is the greatest stimulus that can be given to the
spiritual Samskaras. "One moment of company with the holy makes a ship
to cross this ocean of life." Such is the power of association. So this
repetition of Om, and thinking of its meaning, is keeping good company
in your own mind. Study, and then meditate on what you have studied.
Thus light will come to you, the Self will become manifest. \\

But one must think of Om, and of its meaning too. Avoid evil
company, because the scars of old wounds are in you, and evil company
is just the thing that is necessary to call them out. In the same way
we are told that good company will call out the good impressions that
are in us, but which have become latent. There is nothing holier in the
world than to keep good company, because the good impressions will then
tend to come to the surface. \\

\begin{center}
\begin{sanskrit}
ततः प्रत्यक्चेतनाधिगमोऽप्यन्तरायाभावश्च ॥२९॥
\end{sanskrit}
\end{center}
29. From that is gained (the knowledge of) introspection, and
the destruction of obstacles. \\

The first manifestation of the repetition and thinking of Om
is that the introspective power will manifest more and more, all the
mental and physical obstacles will begin to vanish. What are the
obstacles to the Yogi? \\

\begin{center}
\begin{sanskrit}
व्याधि-स्त्यान-संशय-प्रमादालस्याविरति-भ्रान्तिदर्शनालब्धभूमिकत्वानवस्थितत्वानि
चित्तविक्षेपास्तेऽन्तरायाः ॥३०॥
\end{sanskrit}
\end{center}
30. Disease, mental laziness, doubt, lack of enthusiasm,
lethargy, clinging to sense-enjoyments, false perception, non-attaining
concentration, and falling away from the state when obtained, are the
obstructing distractions.\\

None\\

\begin{center}
\begin{sanskrit}
दुःख-दौर्मनस्याङ्गमेजयत्व-श्वासप्रश्वासा
विक्षेपसहभुवः ॥३१॥
\end{sanskrit}
\end{center}
31. Grief, mental distress, tremor of the body, irregular
breathing, accompany non-retention of concentration. \\

Concentration will bring perfect repose to mind and body every
time it is practised. When the practice has been misdirected, or not
enough controlled, these disturbances come. Repetition of Om and
self-surrender to the Lord will strengthen the mind, and bring fresh
energy. The nervous shakings will come to almost everyone. Do
not mind them at all, but keep on practising. Practice will cure them
and make the seat firm. \\

\begin{center}
\begin{sanskrit}
तत्प्रतिषेधार्थमेकतत्त्वाभ्यासः ॥३२॥
\end{sanskrit}
\end{center}
32. To remedy this, the practice of one subject (should be
made). \\

Making the mind take the form of one object for some time will
destroy these obstacles. This is general advice. In the following
aphorisms it will be expanded and particularized. As one practice
cannot suit everyone, various methods will be advanced, and everyone by
actual experience will find out that which helps him most. \\

\begin{center}
\begin{sanskrit}
None
\end{sanskrit}
\end{center}
33. Friendship, mercy, gladness, and indifference, being
thought of in regard to subjects, happy, unhappy, good, and evil
respectively, pacify the Chitta. \\

We must have these four sorts of ideas. We must have
friendship for all; we must be merciful towards those that are in
misery; when people are happy, we ought to be happy; and to the wicked
we must be indifferent. So with all subjects that come before us. If
the subject is a good one, we shall feel friendly towards it; if the
subject of thought is one that is miserable, we must be merciful
towards it. If it is good, we must be glad; if it is evil, we must be
indifferent. These attitudes of the mind towards the different subjects
that come before it will make the mind peaceful. Most of our
difficulties in our daily lives come from being unable to hold our
minds in this way. For instance, if a man does evil to us, instantly we
want to react evil, and every reaction of evil shows that we are not
able to hold the Chitta down; it comes out in waves towards the object,
and we lose our power. Every reaction in the form of hatred or evil is
so much loss to the mind; and
every evil thought or deed of hatred, or any thought of reaction, if it
is controlled, will be laid in our favour. It is not that we lose by
thus restraining ourselves; we are gaining infinitely more than we
suspect. Each time we suppress hatred, or a feeling of anger, it is so
much good energy stored up in our favour; that piece of energy will be
converted into the higher powers.\\

\begin{center}
\begin{sanskrit}
प्रच्छर्दन-विधारणाभ्यां वा प्राणस्य ॥३४॥
\end{sanskrit}
\end{center}
34. By throwing out and restraining the Breath. \\

The word used is Prâna. Prana is not exactly breath. It is the
name for the energy that is in the universe. Whatever you see in the
universe, whatever moves or works, or has life, is a manifestation of
this Prana. The sum-total of the energy displayed in the universe is
called Prana. This Prana, before a cycle begins, remains in an almost
motionless state; and when the cycle begins, this Prana begins to
manifest itself. It is this Prana that is manifested as motion — as the
nervous motion in human beings or animals; and the same Prana is
manifesting as thought, and so on. The whole universe is a combination
of Prana and Âkâsha; so is the human body. Out of Akasha you get the
different materials that you feel and see, and out of Prana all the
various forces. Now this throwing out and restraining the Prana is what
is called Pranayama. Patanjali, the father of the Yoga philosophy, does
not give very many particular directions about Pranayama, but later on
other Yogis found out various things about this Pranayama, and made of
it a great science. With Patanjali it is one of the many ways, but he
does not lay much stress on it. He means that you simply throw the air
out, and draw it in, and hold it for some time, that is all, and by
that, the mind will become a little calmer. But, later on, you will
find that out of this is evolved a particular science called Pranayama.
We shall hear a little of what these later Yogis have to say. \\

Some of this I have told you before, but a little repetition
will serve to fix it in your minds. First, you must remember that this
Prana is not the breath; but that which causes the motion of the
breath, that which is the vitality of the breath, is the Prana. Again,
the word Prana is used for all the senses; they are all called Pranas,
the mind is called Prana; and so we see that Prana is force. And yet we
cannot call it force, because force is only the manifestation of it. It
is that which manifests itself as force and everything else in the way
of motion. The Chitta, the mind-stuff, is the engine which draws in the
Prana from the surroundings, and manufactures out of Prana the various
vital forces — those that keep the body in preservation — and thought,
will, and all other powers. By the abovementioned process of breathing
we can control all the various motions in the body, and the various
nerve currents that are running through the body. First we begin to
recognise them, and then we slowly get control over them. \\

Now, these later Yogis consider that there are three main
currents of this Prana in the human body. One they call Idâ, another
Pingalâ, and the third Sushumnâ. Pingala, according to them, is on the
right side of the spinal column, and the Ida on the left, and in the
middle of the spinal column is the Sushumna, an empty channel. Ida and
Pingala, according to them, are the currents working in every man, and
through these currents, we are performing all the functions of life.
Sushumna is present in all, as a possibility; but it works only in the
Yogi. You must remember that Yoga changes the body. As you go on
practising, your body changes; it is not the same body that you had
before the practice. That is very rational, and can be explained,
because every new thought that we have must make, as it were, a new
channel through the brain, and that explains the tremendous
conservatism of human nature. Human nature likes to run through the
ruts that are already there, because it is easy. If we think, just for
example's sake, that the mind is like a needle, and the brain substance
a soft lump before it, then each thought that we have makes a street,
as it were, in the brain, and this street would close up, but for the
grey matter which comes and makes a lining to keep it separate. If
there were no grey matter, there would be no memory, because memory
means going over these old streets, retracing a thought as it were. Now
perhaps you have marked that when one talks on subjects in which one
takes a few ideas that are familiar to everyone, and combines and
recombines them, it is easy to follow because these channels are
present in everyone's brain, and it is only necessary to recur to them.
But whenever a new subject comes, new channels have to be made, so it
is not understood readily. And that is why the brain (it is the brain,
and not the people themselves) refuses unconsciously to be acted upon
by new ideas. It resists. The Prana is trying to make new channels, and
the brain will not allow it. This is the secret of conservatism. The
fewer channels there have been in the brain, and the less the needle of
the Prana has made these passages, the more conservative will be the
brain, the more it will struggle against new thoughts. The more
thoughtful the man, the more complicated will be the streets in his
brain, and the more easily he will take to new ideas, and understand
them. So with every fresh idea, we make a new impression in the brain,
cut new channels through the brain-stuff, and that is why we find that
in the practice of Yoga (it being an entirely new set of thoughts and
motives) there is so much physical resistance at first. That is why we
find that the part of religion which deals with the world-side of
nature is so widely accepted, while the other part, the philosophy, or
the psychology, which clears with the inner nature of man, is so
frequently neglected. \\

We must remember the definition of this world of ours; it is
only the Infinite Existence projected into the
plane of consciousness. A little of the Infinite is projected into
consciousness, and that we call our world. So there is an Infinite
beyond; and religion has to deal with both — with the little lump we
call our world, and with the Infinite beyond. Any religion which deals
with one only of these two will be defective. It must deal with both.
The part of religion which deals with the part of the Infinite which
has come into the plane of consciousness, got itself caught, as it
were, in the plane of consciousness, in the cage of time, space, and
causation, is quite familiar to us, because we are in that already, and
ideas about this world have been with us almost from time immemorial.
The part of religion which deals with the Infinite beyond comes
entirely new to us, and getting ideas about it produces new channels in
the brain, disturbing the whole system, and that is why you find in the
practice of Yoga ordinary people are at first turned out of their
grooves. In order to lessen these disturbances as much as possible, all
these methods are devised by Patanjali, that we may practice any one of
them best suited to us. \\

\begin{center}
\begin{sanskrit}
विषयवती वा प्रवृत्तिरुत्पन्ना मनसः
स्थितिनिबन्धिनी ॥३५॥
\end{sanskrit}
\end{center}
35. Those forms of concentration that bring extraordinary
sense-perceptions cause perseverance of the mind. \\

This naturally comes with Dhâranâ, concentration; the Yogis
say, if the mind becomes concentrated on the tip of the nose, one
begins to smell, after a few days, wonderful perfumes. If it becomes
concentrated at the root of the tongue, one begins to hear sounds; if
on the tip of the tongue, one begins to taste wonderful flavours; if on
the middle of the tongue, one feels as if one were coming in contact
with something. If one concentrates one's mind on the palate, one
begins to see peculiar things. If a man whose mind is disturbed wants
to take up some of these practices of Yoga, yet doubts the truth of
them, he will have his doubts
set at rest when, after a little practice, these things come to him,
and he will persevere. \\

\begin{center}
\begin{sanskrit}
None
\end{sanskrit}
\end{center}
36. Or (by the meditation on) the Effulgent Light, which is
beyond all sorrow. \\

This is another sort of concentration. Think of the lotus of
the heart, with petals downwards, and running through it, the Sushumna;
take in the breath, and while throwing the breath out imagine that the
lotus is turned with the petals upwards, and inside that lotus is an
effulgent light. Meditate on that. \\

\begin{center}
\begin{sanskrit}
वीतरागविषयं वा चित्तम् ॥३७॥
\end{sanskrit}
\end{center}
37. Or (by meditation on) the heart that has given up all
attachment to sense-objects. \\

Take some holy person, some great person whom you revere, some
saint whom you know to be perfectly nonattached, and think of his
heart. That heart has become non-attached, and meditate on that heart;
it will calm the mind. If you cannot do that, there is the next way:\\

\begin{center}
\begin{sanskrit}
स्वप्ननिद्राज्ञानालम्बनं वा ॥३८॥
\end{sanskrit}
\end{center}
38. Or by meditating on the knowledge that comes in sleep. \\

Sometimes a man dreams that he has seen angels coming to him
and talking to him, that he is in an ecstatic condition, that he has
heard music floating through the air. He is in a blissful condition in
that dream, and when he wakes, it makes a deep impression on him. Think
of that dream as real, and meditate upon it. If you cannot do that,
meditate on any holy thing that pleases you. \\

\begin{center}
\begin{sanskrit}
यथाभिमतध्यानाद्वा ॥३९॥
\end{sanskrit}
\end{center}
39. Or by the meditation on anything that appeals to one as
good. \\

This does not mean any wicked subject, but anything good that
you like, any place that you like best, any scenery that you like best,
any idea that you like best, anything that will concentrate the mind. \\

\begin{center}
\begin{sanskrit}
परमाणु परममहत्त्वान्तोऽस्य वशीकारः ॥४०॥
\end{sanskrit}
\end{center}
40. The Yogi's mind thus meditating, becomes unobstructed from
the atomic to the infinite. \\

The mind, by this practice, easily contemplates the most
minute, as well as the biggest thing. Thus the mindwaves become
fainter. \\

\begin{center}
\begin{sanskrit}
क्षीणवृत्तेरभिजातस्येव
मणेर्ग्रहीतृ-ग्रहण-ग्राह्येषु
तत्स्थ-तदञ्जनता समापत्तिः ॥४१॥
\end{sanskrit}
\end{center}
41. The Yogi whose Vrittis have thus become powerless
(controlled) obtains in the receiver, (the instrument of) receiving,
and the received (the Self, the mind, and external objects),
concentratedness arid sameness like the crystal (before different
coloured objects). \\

What results from this constant meditation? We must remember
how in a previous aphorism Patanjali went into the various states of
meditation, how the first would be the gross, the second the fine, and
from them the advance was to still finer objects. The result of these
meditations is that we can meditate as easily on the fine as on the
gross objects. Here the Yogi sees the three things, the receiver, the
received, and the receiving instrument, corresponding to the Soul,
external objects, and the mind. There are three objects of meditation
given us. First, the gross things, as bodies, or material objects;
second, fine things, as the mind, the Chitta; and third, the Purusha
qualified, not the Purusha itself, but the Egoism. By practice, the
Yogi gets established in all these meditations. Whenever he meditates
he can keep out all other
thoughts; he becomes identified with that on which he meditates. When
he meditates, he is like a piece of crystal. Before flowers the crystal
becomes almost identified with the flowers. If the flower is red, the
crystal looks red, or if the flower is blue, the crystal looks blue. \\

\begin{center}
\begin{sanskrit}
तत्र शब्दार्थज्ञानविकल्पैः सङ्कीर्णा
सवितर्का समापत्तिः ॥४२॥
\end{sanskrit}
\end{center}
42. Sound, meaning, and resulting knowledge, being mixed up,
is (called) Samadhi with question. \\

Sound here means vibration, meaning the nerve currents which
conduct it; and knowledge, reaction. All the various meditations we
have had so far, Patanjali calls Savitarka (meditation with question).
Later on he gives us higher and higher Dhyânas. In these that are
called "with question," we keep the duality of subject and object,
which results from the mixture of word, meaning, and knowledge. There
is first the external vibration, the word. This, carried inward by the
sense currents, is the meaning. After that there comes a reactionary
wave in the Chitta, which is knowledge, but the mixture of these three
makes up what we call knowledge. In all the meditations up to this we
get this mixture as objects of meditation. The next Samadhi is higher. \\

\begin{center}
\begin{sanskrit}
None
\end{sanskrit}
\end{center}
स्मृतिपरिशुद्धौ स्वरूपशून्येवार्थमात्रनिर्भासा निर्वितर्का ॥४३॥\\

43. The Samadhi called "without question" (comes) when the
memory is purified, or devoid of qualities, expressing only the meaning
(of the meditated object). \\

It is by the practice of meditation of these three that we
come to the state where these three do not mix. We can get rid of them.
We will first try to understand what these three are. Here is the
Chitta; you will always remember
the simile of the mind-stuff to a lake, and the vibration, the word,
the sound, like a pulsation coming over it. You have that calm lake in
you, and I pronounce a word, "Cow". As soon as it enters through your
ears there is a wave produced in your Chitta along with it. So that
wave represents the idea of the cow, the form or the meaning as we call
it. The apparent cow that you know is really the wave in the mind-stuff
that comes as a reaction to the internal and external sound vibrations.
With the sound, the wave dies away; it can never exist without a word.
You may ask how it is, when we only think of the cow, and do not hear a
sound. You make that sound yourself. You are saying "cow" faintly in
your mind, and with that comes a wave. There cannot be any wave without
this impulse of sound; and when it is not from outside, it is from
inside, and when the sound dies, the wave dies. What remains? The
result of the reaction, and that is knowledge. These three are so
closely combined in our mind that we cannot separate them. When the
sound comes, the senses vibrate, and the wave rises in reaction; they
follow so closely upon one another that there is no discerning one from
the other. When this meditation has been practiced for a long time,
memory, the receptacle of all impressions, becomes purified, and we are
able clearly to distinguish them from one another. This is called
Nirvitarka, concentration without question. \\

\begin{center}
\begin{sanskrit}
एतयैव सविचारा निर्विचारा च सूक्ष्मविषया
व्याख्याता ॥४४॥
\end{sanskrit}
\end{center}
44. By this process (the concentrations) with discrimination
and without discrimination, whose objects are finer, are (also)
explained. \\

A process similar to the preceding is applied again; only, the
objects to be taken up in the former meditations are gross; in this
they are fine. \\

\begin{center}
\begin{sanskrit}
सूक्ष्मविषयत्वञ्चालिङ्ग-पर्यवसानम् ॥४५॥
\end{sanskrit}
\end{center}
45. The finer objects end with the Pradhâna. \\

The gross objects are only the elements and everything
manufactured out of them. The fine objects begin with the Tanmatras or
fine particles. The organs, the mind, (The mind, or common sensorium,
the aggregate of all the senses), egoism, the mind-stuff (the cause of
all manifestation), the equilibrium state of Sattva, Rajas, and Tamas
materials — called Pradhâna (chief), Prakriti (nature), or Avyakta
(unmanifest) — are all included within the category of fine objects,
the Purusha (the Soul) alone being excepted. \\

\begin{center}
\begin{sanskrit}
ता एव सबीजः समाधिः ॥४६॥
\end{sanskrit}
\end{center}
46. These concentrations are with seed. \\

These do not destroy the seeds of past actions, and thus
cannot give liberation, but what they bring to the Yogi is stated in
the following aphorism. \\

\begin{center}
\begin{sanskrit}
निर्विचार-वैशारद्येऽध्यात्मप्रसादः ॥४७॥
\end{sanskrit}
\end{center}
47. The concentration "without discrimination" being purified,
the Chitta becomes firmly fixed. \\

\begin{center}
\begin{sanskrit}
ऋतम्भरा तत्र प्रज्ञा ॥४८॥
\end{sanskrit}
\end{center}
48. The knowledge in that is called "filled with Truth". \\

The next aphorism will explain this. \\

\begin{center}
\begin{sanskrit}
श्रुतानुमानप्रज्ञाभ्यामन्यविषया
विशेषार्थत्वात् ॥४९॥
\end{sanskrit}
\end{center}
49. The knowledge that is gained from testimony and inference
is about common objects. That from the Samadhi just mentioned is of a
much higher order, being able to penetrate where inference and
testimony cannot go. \\

The idea is that we have to get our knowledge of ordinary
objects by direct perception, and by inference therefrom, and from
testimony of people who are competent. By "people who are competent,"
the Yogis always mean the Rishis, or the Seers of the thoughts recorded
in the scriptures — the Vedas. According to them, the only proof of the
scriptures is that they were the testimony of competent persons, yet
they say the scriptures cannot take us to realisation. We can read all
the Vedas, and yet will not realise anything, but when we practise
their teachings, then we attain to that state which realises what the
scriptures say, which penetrates where neither reason nor perception
nor inference can go, and where the testimony of others cannot avail.
This is what is meant by the aphorism. \\

Realisation is real religion, all the rest is only preparation
— hearing lectures, or reading books, or reasoning is merely preparing
the ground; it is not religion. Intellectual assent and intellectual
dissent are not religion. The central idea of the Yogis is that just as
we come in direct contact with objects of the senses, so religion even
can be directly perceived in a far more intense sense. The truths of
religion, as God and Soul, cannot be perceived by the external senses.
I cannot see God with my eyes, nor can I touch Him with my hands, and
we also know that neither can we reason beyond the senses. Reason
leaves us at a point quite indecisive; we may reason all our lives, as
the world has been doing for thousands of years, and the result is that
we find we are incompetent to prove or disprove the facts of religion.
What we perceive directly we take as the basis, and upon that basis we
reason. So it is obvious that reasoning has to run within these bounds
of perception. It can never go beyond. The whole scope of realisation,
therefore, is beyond sense-perception. The Yogis say that man can go
beyond his direct sense-perception, and beyond his reason also. Man has
in him the faculty, the power,
of transcending his intellect even, a power which is in every being,
every creature. By the practice of Yoga that power is aroused, and then
man transcends the ordinary limits of reason, and directly perceives
things which are beyond all reason. \\

\begin{center}
\begin{sanskrit}
तज्जः संस्कारोऽन्यसंस्कारप्रतिबन्धी ॥५०॥
\end{sanskrit}
\end{center}
50. The resulting impression from this Samadhi obstructs all
other impressions. \\

We have seen in the foregoing aphorism that the only way of
attaining to that superconsciousness is by concentration, and we have
also seen that what hinder the mind from concentration are the past
Samskaras, impressions. All of you have observed that, when you are
trying to concentrate your mind, your thoughts wander. When you are
trying to think of God, that is the very time these Samskaras appear.
At other times they are not so active; but when you want them not, they
are sure to be there, trying their best to crowd in your mind. Why
should that be so? Why should they be much more potent at the time of
concentration? It is because you are repressing them, and they react
with all their force. At other times they do not react. How countless
these old past impressions must be, all lodged somewhere in the Chitta,
ready, waiting like tigers, to jump up! These have to be suppressed
that the one idea which we want may arise, to the exclusion of the
others. Instead they are all struggling to come up at the same time.
These are the various powers of the Samskaras in hindering
concentration of the mind. So this Samadhi which has just been given is
the best to be practised, on account of its power of suppressing the
Samskaras. The Samskara which will be raised by this sort of
concentration will be so powerful that it will hinder the action of the
others, and hold them in check. \\

\begin{center}
\begin{sanskrit}
तस्यापि निरोधे सर्वनिरोधान्निर्बीजः समाधिः
॥५१॥
\end{sanskrit}
\end{center}
51. By the restraint of even this (impression, which obstructs
all other impressions), all being restrained, comes the "seedless"
Samadhi. \\

You remember that our goal is to perceive the Soul itself. We
cannot perceive the Soul, because it has got mingled up with nature,
with the mind, with the body. The ignorant man thinks his body is the
Soul. The learned man thinks his mind is the Soul. But both of them are
mistaken. What makes the Soul get mingled up with all this? Different
waves in the Chitta rise and cover the Soul; we only see a little
reflection of the Soul through these waves; so, if the wave is one of
anger, we see the Soul as angry; "I am angry," one says. If it is one
of love, we see ourselves reflected in that wave, and say we are
loving. If that wave is one of weakness, and the Soul is reflected in
it, we think we are weak. These various ideas come from these
impressions, these Samskaras covering the Soul. The real nature of the
Soul is not perceived as long as there is one single wave in the lake
of the Chitta; this real nature will never be perceived until all the
waves have subsided. So, first, Patanjali teaches us the meaning of
these waves; secondly, the best way to repress them; and thirdly, how
to make one wave so strong as to suppress all other waves, fire eating
fire as it were. When only one remains, it will be easy to suppress
that also, and when that is gone, this Samadhi or concentration is
called seedless. It leaves nothing, and the Soul is manifested just as
It is, in Its own glory. Then alone we know that the Soul is not a
compound; It is the only eternal simple in the universe, and as such,
It cannot be born, It cannot die; It is immortal, indestructible, the
ever-living essence of intelligence. \\

\section{CONCENTRATION: ITS SPIRITUAL USES }
\begin{center}\textit{PATANJALI'S YOGA APHORISMS}\end{center}

\begin{center}\textit{CHAPTER I}\end{center}

\begin{center}
\begin{sanskrit}
अथ योगानुशासनम् ॥१॥
\end{sanskrit}
\end{center}
1. Now concentration is explained. \\

\begin{center}
\begin{sanskrit}
योगश्चित्तवृत्तिनिरोधः ॥२॥
\end{sanskrit}
\end{center}
2. Yoga is restraining the mind-stuff (Chitta) from taking
various forms (Vrittis). \\

A good deal of explanation is necessary here. We have to
understand what Chitta is, and what the Vrittis are. I have eyes. Eyes
do not see. Take away the brain centre which is in the head, the eyes
will still be there, the retinae complete, as also the pictures of
objects on them, and yet the eyes will not see. So the eyes are only a
secondary instrument, not the organ of vision. The organ of vision is
in a nerve centre of the brain. The two eyes will not be sufficient.
Sometimes a man is asleep with his eyes open. The light is there and
the picture is there, but a third thing is necessary — the mind must be
joined to the organ. The eye is the external instrument; we need also
the brain centre and the agency of the mind. Carriages roll down a
street, and you do not hear them. Why? Because your mind has not
attached itself to the organ of hearing. First, there is the
instrument, then there is the organ, and third, the mind attached to
these two. The mind takes the impression farther in, and presents it to
the determinative faculty — Buddhi — which reacts. Along with this
reaction flashes the idea of egoism. Then this mixture of action and
reaction is presented to the
Purusha, the real Soul, who perceives an object in this mixture. The
organs (Indriyas), together with the mind (Manas), the determinative
faculty (Buddhi), and egoism (Ahamkâra), form the group called the
Antahkarana (the internal instrument). They are but various processes
in the mind-stuff, called Chitta. The waves of thought in the Chitta
are called Vrittis (literally "whirlpool") . What is thought? Thought
is a force, as is gravitation or repulsion. From the infinite
storehouse of force in nature, the instrument called Chitta takes hold
of some, absorbs it and sends it out as thought. Force is supplied to
us through food, and out of that food the body obtains the power of
motion etc. Others, the finer forces, it throws out in what we call
thought. So we see that the mind is not intelligent; yet it appears to
be intelligent. Why? Because the intelligent soul is behind it. You are
the only sentient being; mind is only the instrument through which you
catch the external world. Take this book; as a book it does not exist
outside, what exists outside is unknown and unknowable. The unknowable
furnishes the suggestion that gives a blow to the mind, and the mind
gives out the reaction in the form of a book, in the same manner as
when a stone is thrown into the water, the water is thrown against it
in the form of waves. The real universe is the occasion of the reaction
of the mind. A book form, or an elephant form, or a man form, is not
outside; all that we know is our mental reaction from the outer
suggestion. "Matter is the permanent possibility of sensations," said
John Stuart Mill. It is only the suggestion that is outside. Take an
oyster for example. You know how pearls are made. A parasite gets
inside the shell and causes irritation, and the oyster throws a sort of
enamelling round it, and this makes the pearl. The universe of
experience is our own enamel, so to say, and the real universe is the
parasite serving as nucleus. The ordinary man will never understand it,
because when he tries to do so, he throws out
an enamel, and sees only his own enamel. Now we understand what is
meant by these Vrittis. The real man is behind the mind; the mind is
the instrument his hands; it is his intelligence that is percolating
through the mind. It is only when you stand behind the mind that it
becomes intelligent. When man gives it up, it falls to pieces and is
nothing. Thus you understand what is meant by Chitta. It is the
mind-stuff, and Vrittis are the waves and ripples rising in it when
external causes impinge on it. These Vrittis are our universe. \\

The bottom of a lake we cannot see, because its surface is
covered with ripples. It is only possible for us to catch a glimpse of
the bottom, when the ripples have subsided, and the water is calm. If
the water is muddy or is agitated all the time, the bottom will not be
seen. If it is clear, and there are no waves, we shall see the bottom.
The bottom of the lake is our own true Self; the lake is the Chitta and
the waves the Vrittis. Again, the mind is in three states, one of which
is darkness, called Tamas, found in brutes and idiots; it only acts to
injure. No other idea comes into that state of mind. Then there is the
active state of mind, Rajas, whose chief motives are power and
enjoyment. "I will be powerful and rule others." Then there is the
state called Sattva, serenity, calmness, in which the waves cease, and
the water of the mind-lake becomes clear. It is not inactive, but
rather intensely active. It is the greatest manifestation of power to
be calm. It is easy to be active. Let the reins go, and the horses will
run away with you. Anyone can do that, but he who can stop the plunging
horses is the strong man. Which requires the greater strength, letting
go or restraining? The calm man is not the man who is dull. You must
not mistake Sattva for dullness or laziness. The calm man is the one
who has control over the mind waves. Activity is the manifestation of
inferior strength, calmness, of the superior. \\

The Chitta is always trying to get back to its natural pure
state, but the organs draw it out. To restrain it, to check this
outward tendency, and to start it on the return journey to the essence
of intelligence is the first step in Yoga, because only in this way can
the Chitta get into its proper course. \\

Although the Chitta is in every animal, from the lowest to the
highest, it is only in the human form that we find it as the intellect.
Until the mind-stuff can take the form of intellect it is not possible
for it to return through all these steps, and liberate the soul.
Immediate salvation is impossible for the cow or the dog, although they
have mind, because their Chitta cannot as yet take that form which we
call intellect. \\

The Chitta manifests itself in the following forms —
scattering, darkening, gathering, one-pointed, and concentrated. The
scattering form is activity. Its tendency is to manifest in the form of
pleasure or of pain. The darkening form is dullness which tends to
injury. The commentator says, the third form is natural to the Devas,
the angels, and the first and second to the demons. The gathering form
is when it struggles to centre itself. The one-pointed form is when it
tries to concentrate, and the concentrated form is what brings us to
Samâdhi. \\

\begin{center}
\begin{sanskrit}
तदा द्रष्टुः स्वरूपेऽवस्थानम् ॥३॥
\end{sanskrit}
\end{center}
3. At that time (the time of concentration) the seer (Purusha)
rests in his own (unmodified) state. \\

As soon as the waves have stopped, and the lake has become
quiet, we see its bottom. So with the mind; when it is calm, we see
what our own nature is; we do not mix ourselves but remain our own
selves. \\

\begin{center}
\begin{sanskrit}
वृत्तिसारूप्यमितरत्र ॥४॥
\end{sanskrit}
\end{center}
4. At other times (other than that of concentration) the seer
is identified with the modifications. \\

For instance, someone blames me; this produces a modification,
Vritti, in my mind, and I identify myself with it and the result is
misery. \\

\begin{center}
\begin{sanskrit}
वृत्तयः पंचतय्यः क्लिष्टा अक्लिष्टाः ॥५॥
\end{sanskrit}
\end{center}
5. There are five classes of modifications, (some) painful and
(others) not painful. \\

\begin{center}
\begin{sanskrit}
प्रमाण-विपर्यय-विकल्प-निद्रा-स्मृतयः ॥६॥
\end{sanskrit}
\end{center}
6. (These are) right knowledge, indiscrimination, verbal
delusion, sleep, and memory. \\

\begin{center}
\begin{sanskrit}
प्रत्यक्षानुमानागमाः प्रमाणानि ॥७॥
\end{sanskrit}
\end{center}
7. Direct perception, inference, and
competent evidence are
proofs. \\

When two of our perceptions do not contradict each other, we
call it proof. I hear something, and if it contradicts something
already perceived, I begin to fight it out, and do not believe it.
There are also three kinds of proof. Pratyaksha, direct perception;
whatever we see and feel, is proof, if there has been nothing to delude
the senses. I see the world; that is sufficient proof that it exists.
Secondly, Anumâna, inference; you see a sign, and from the sign you
come to the thing signified. Thirdly, Âptavâkya, the direct evidence of
the Yogis, of those who have seen the truth. We are all of us
struggling towards knowledge. But you and I have to struggle hard, and
come to knowledge through a long tedious process of reasoning, but the
Yogi, the pure one, has gone beyond all this. Before his mind, the
past, the present, and the future are alike, one book for him to read;
he does not require to go through the tedious processes for knowledge
we have to; his words are proof, because he sees knowledge in himself.
These, for instance, are the authors of the sacred scriptures;
therefore the scriptures are proof. If any such persons are living now
their words will be proof. Other
philosophers go into long discussions about Aptavakya and they say,
"What is the proof of their words?" The proof is their direct
perception. Because whatever I see is proof, and whatever you see is
proof, if it does not contradict any past knowledge. There is knowledge
beyond the senses, and whenever it does not contradict reason and past
human experience, that knowledge is proof. Any madman may come into
this room and say he sees angels around him; that would not be proof.
In the first place, it must be true knowledge, and secondly, it must
not contradict past knowledge, and thirdly, it must depend upon the
character of the man who gives it out. I hear it said that the
character of the man is not of so much importance as what he may say;
we must first hear what he says. This may be true in other things. A
man may be wicked, and yet make an astronomical discovery, but in
religion it is different, because no impure man will ever have the
power to reach the truths of religion. Therefore we have first of all
to see that the man who declares himself to be an Âpta is a perfectly
unselfish and holy person; secondly, that he has reached beyond the
senses; and thirdly, that what he says does not contradict the past
knowledge of humanity. Any new discovery of truth does not contradict
the past truth, but fits into it. And fourthly, that truth must have a
possibility of verification. If a man says, "I have seen a vision," and
tells me that I have no right to see it, I believe him not. Everyone
must have the power to see it for himself. No one who sells his
knowledge is an Apta. All these conditions must be fulfilled; you must
first see that the man is pure, and that he has no selfish motive; that
he has no thirst for gain or fame. Secondly, he must show that he is
superconscious. He must give us something that we cannot get from our
senses, and which is for the benefit of the world. Thirdly, we must see
that it does not contradict other truths; if it contradicts other
scientific truths reject it at once. Fourthly, the man
should never be singular; he should only represent what all men can
attain. The three sorts of proof are, then, direct sense-perception,
inference, and the words of an Apta. I cannot translate this word into
English. It is not the word "inspired", because inspiration is believed
to come from outside, while this knowledge comes from the man himself.
The literal meaning is "attained". \\

\begin{center}
\begin{sanskrit}
विपर्ययो मिथ्याज्ञानमतद्रूपप्रतिष्ठम् ॥८॥
\end{sanskrit}
\end{center}
8. Indiscrimination is false knowledge not established in real
nature. \\

The next class of Vrittis that arises is mistaking one thing
for another, as a piece of mother-of-pearl is taken for a piece of
silver. \\

\begin{center}
\begin{sanskrit}
शब्दज्ञानानुपाती वस्तुशून्यो विकल्पः ॥९॥
\end{sanskrit}
\end{center}
9. Verbal delusion follows from words having no
(corresponding) reality. \\

There is another class of Vrittis called Vikalpa. A word is
uttered, and we do not wait to consider its meaning; we jump to a
conclusion immediately. It is the sign of weakness of the Chitta. Now
you can understand the theory of restraint. The weaker the man, the
less he has of restraint. Examine yourselves always by that test. When
you are going to be angry or miserable, reason it out how it is that
some news that has come to you is throwing your mind into Vrittis. \\

\begin{center}
\begin{sanskrit}
अभाव-प्रत्ययालम्बना-वृत्तिर्निद्रा ॥१०॥
\end{sanskrit}
\end{center}
10. Sleep is a Vritti which embraces the feeling of voidness. \\

The next class of Vrittis is called sleep and dream. When we
awake, we know that we have been sleeping; we can only have memory of
perception. That which we do not perceive we never can have any memory
of. Every reaction is a wave in
the lake. Now, if, during sleep, the mind had no waves, it would have
no perceptions, positive or negative, and, therefore, we would not
remember them. The very reason of our remembering sleep is that during
sleep there was a certain class of waves in the mind. Memory is another
class of Vrittis which is called Smriti. \\

\begin{center}
\begin{sanskrit}
अनुभूतविषयासम्प्रमोषः स्मृतिः ॥११॥
\end{sanskrit}
\end{center}
11. Memory is when the (Vrittis of) perceived subjects do not
slip away (and through impressions come back to consciousness). \\

Memory can come from direct perception, false knowledge,
verbal delusion, and sleep. For instance, you hear a word. That word is
like a stone thrown into the lake of the Chitta; it causes a ripple,
and that ripple rouses a series of ripples; this is memory. So in
sleep. When the peculiar kind of ripple called sleep throws the Chitta
into a ripple of memory, it is called a dream. Dream is another form of
the ripple which in the waking state is called memory.\\

\begin{center}
\begin{sanskrit}
अभ्यासवैराग्याभ्यां तन्निरोधः ॥१२॥
\end{sanskrit}
\end{center}
12. Their control is by practice and nonattachment. \\

The mind, to have non-attachment, must be clear, good, and
rational. Why should we practice? Because each action is like the
pulsations quivering over the surface of the lake. The vibration dies
out, and what is left? The Samskâras, the impressions. When a large
number of these impressions are left on the mind, they coalesce and
become a habit. It is said, "Habit is second nature", it is first
nature also, and the whole nature of man; everything that we are is the
result of habit. That gives us consolation, because, if it is only
habit, we can make and unmake it at any time. The Samskaras are left by
these vibrations passing out of
our mind, each one of them leaving its result. Our character is the
sum-total of these marks, and according as some particular wave
prevails one takes that tone. If good prevails, one becomes good; if
wickedness, one becomes wicked; if joyfulness, one becomes happy. The
only remedy for bad habits is counter habits; all the bad habits that
have left their impressions are to be controlled by good habits. Go on
doing good, thinking holy thoughts continuously; that is the only way
to suppress base impressions. Never say any man is hopeless, because he
only represents a character, a bundle of habits, which can be checked
by new and better ones. Character is repeated habits, and repeated
habits alone can reform character. \\

\begin{center}
\begin{sanskrit}
तत्र स्थितौ यत्नोऽभ्यासः ॥१३॥
\end{sanskrit}
\end{center}
13. Continuous struggle to keep them (the Vrittis) perfectly
restrained is practice. \\

What is practice? The attempt to restrain the mind in Chitta
form, to prevent its going out into waves. \\

\begin{center}
\begin{sanskrit}
स तु दीर्घकालनैरन्तर्यसत्कारासेवितो दृढभूमिः
॥१४॥
\end{sanskrit}
\end{center}
14. It becomes firmly grounded by long constant efforts with
great love (for the end to be attained). \\

Restraint does not come in one day, but by long continued
practice. \\

\begin{center}
\begin{sanskrit}
दृष्टानुश्रविकविषयवितृष्णस्य वशीकारसंज्ञा
वैराग्यम् ॥१५॥
\end{sanskrit}
\end{center}
15. That effect which comes to these who have given up their
thirst after objects, either seen or heard, and which wills to control
the objects, is non-attachment. \\

The two motive powers of our actions are (1) what we see
ourselves, (2) the experience of others. These two
forces throw the mind, the lake, into various waves. Renunciation is
the power of battling against these forces and holding the mind in
check. Their renunciation is what see want. I am passing through a
street, and a man comes and takes away my watch. That is my own
experience. I see it myself, and it immediately throws my Chitta into a
wave, taking the form of anger. Allow not that to come. If you cannot
prevent that, you are nothing; if you can, you have Vairâgya. Again,
the experience of the worldly-minded teaches us that sense-enjoyments
are the highest ideal. These are tremendous temptations. To deny them,
and not allow the mind to come to a wave form with regard to them, is
renunciation; to control the twofold motive powers arising from my own
experience and from the experience of others, and thus prevent the
Chitta from being governed by them, is Vairagya. These should be
controlled by me, and not I by them. This sort of mental strength is
called renunciation. Vairagya is the only way to freedom. \\

\begin{center}
\begin{sanskrit}
तत्परं पुरुषख्यातेर्गुणवैतृष्ण्यम् ॥१६॥
\end{sanskrit}
\end{center}
16. That is extreme non-attachment which gives up even the
qualities, and comes from the knowledge of (the real nature of) the
Purusha. \\

It is the highest manifestation of the power of Vairagya when
it takes away even our attraction towards the qualities. We have first
to understand what the Purusha, the Self, is and what the qualities
are. According to Yoga philosophy, the whole of nature consists of
three qualities or forces; one is called Tamas, another Rajas, and the
third Sattva. These three qualities manifest themselves in the physical
world as darkness or inactivity, attraction or repulsion, and
equilibrium of the two. Everything that is in nature, all
manifestations, are combinations and recombinations of these three
forces. Nature has been divided
into various categories by the Sânkhyas; the Self of man is beyond all
these, beyond nature. It is effulgent, pure, and perfect. Whatever of
intelligence we see in nature is but the reflection of this Self upon
nature. Nature itself is insentient. You must remember that the word
nature also includes the mind; mind is in nature; thought is in nature;
from thought, down to the grossest form of matter, everything is in
nature, the manifestation of nature. This nature has covered the Self
of man, and when nature takes away the covering, the self appears in
Its own glory. The non-attachment, as described in aphorism 15 (as
being control of objects or nature) is the greatest help towards
manifesting the Self. The next aphorism defines Samadhi, perfect
concentration which is the goal of the Yogi. \\

\begin{center}
\begin{sanskrit}
वितर्कविचारानन्दास्मितानुगमात् सम्प्रज्ञातः
॥१७॥
\end{sanskrit}
\end{center}
17. The concentration called right knowledge is that which is
followed by reasoning, discrimination bliss, unqualified egoism. \\

Samadhi is divided into two varieties. One is called the
Samprajnâta, and the other the Asamprajnâta. In the Samprajnata Samadhi
come all the powers of controlling nature. It is of four varieties. The
first variety is called the Savitarka, when the mind meditates upon an
object again and again, by isolating it from other objects. There are
two sorts of objects for meditation in the twenty-five categories of
the Sankhyas, (1) the twenty-four insentient categories of Nature, and
(2) the one sentient Purusha. This part of Yoga is based entirely on
Sankhya philosophy, about which I have already told you. As you will
remember, egoism and will and mind have a common basis, the Chitta or
the mind-stuff, out of which they are all manufactured. The mind-stuff
takes in the forces of nature, and projects them as thought. There must
be something, again, where both force and matter are one.
This is called Avyakta, the unmanifested state of nature before
creation, and to which, after the end of a cycle, the whole of nature
returns, to come out again after another period. Beyond that is the
Purusha, the essence of intelligence. Knowledge is power, and as soon
as we begin to know a thing, we get power over it; so also when the
mind begins to meditate on the different elements, it gains power over
them. That sort of meditation where the external gross elements are the
objects is called Savitarka. Vitarka means question; Savitarka, with
question, questioning the elements, as it were, that they may give
their truths and their powers to the man who meditates upon them. There
is no liberation in getting powers. It is a worldly search after
enjoyments, and there is no enjoyment in this life; all search for
enjoyment is vain; this is the old, old lesson which man finds so hard
to learn. When he does learn it, he gets out of the universe and
becomes free. The possession of what are called occult powers is only
intensifying the world, and in the end, intensifying suffering. Though
as a scientist Patanjali is bound to point out the possibilities of
this science, he never misses an opportunity to warn us against these
powers. \\

Again, in the very same meditation, when one struggles to take
the elements out of time and space, and think of them as they are, it
is called Nirvitarka, without question. When the meditation goes a step
higher, and takes the Tanmatras as its object, and thinks of them as in
time and space, it is called Savichâra, with discrimination; and when
in the same meditation one eliminates time and space, and thinks of the
fine elements as they are, it is called Nirvichâra, without
discrimination. The next step is when the elements are given up, both
gross and fine, and the object of meditation is the interior organ, the
thinking organ. When the thinking organ is thought of as bereft of the
qualities of activity and dullness, it is then called
Sânanda, the blissful Samadhi. When the mind itself is the object of
meditation, when meditation becomes very ripe and concentrated, when
all ideas of the gross and fine materials are given up, when the Sattva
state only of the Ego remains, but differentiated from all other
objects, it is called Sâsmita Samadhi. The man who has attained to this
has attained to what is called in the Vedas "bereft of body". He can
think of himself as without his gross body; but he will have to think
of himself as with a fine body. Those that in this state get merged in
nature without attaining the goal are called Prakritilayas, but those
who do not stop even there reach the goal, which is freedom. \\

\begin{center}
\begin{sanskrit}
विरामप्रत्ययाभ्यासपूर्वः संस्कारशेषोऽन्यः
॥१८॥
\end{sanskrit}
\end{center}
18. There is another Samadhi which is attained by the constant
practice of cessation of all mental activity, in which the Chitta
retains only the unmanifested impressions. \\

This is the perfect superconscious Asamprajnata Samadhi, the
state which gives us freedom. The first state does not give us freedom,
does not liberate the soul. A man may attain to all powers, and yet
fall again. There is no safeguard until the soul goes beyond nature. It
is very difficult to do so, although the method seems easy. The method
is to meditate on the mind itself, and whenever thought comes, to
strike it down, allowing no thought to come into the mind, thus making
it an entire vacuum. When we can really do this, that very moment we
shall attain liberation. When persons without training and preparation
try to make their minds vacant, they are likely to succeed only in
covering themselves with Tamas, the material of ignorance, which makes
the mind dull and stupid, and leads them to think that they are making
a vacuum of the mind. To be able to really do that is to
manifest the greatest strength, the highest control. When this state,
Asamprajnata, superconsciousness, is reached, the Samadhi becomes
seedless. What is meant by that? In a concentration where there is
consciousness, where the mind succeeds only in quelling the waves in
the Chitta and holding them down, the waves remain in the form of
tendencies. These tendencies (or seeds) become waves again, when the
time comes. But when you have destroyed all these tendencies, almost
destroyed the mind, then the Samadhi becomes seedless; there are no
more seeds in the mind out of which to manufacture again and again this
plant of life, this ceaseless round of birth and death. \\

You may ask, what state would that be in which there is no
mind, there is no knowledge? What we call knowledge is a lower state
than the one beyond knowledge. You must always bear in mind that the
extremes look very much alike. If a very low vibration of ether is
taken as darkness, an intermediate state as light, very high vibration
will be darkness again. Similarly, ignorance is the lowest state,
knowledge is the middle state, and beyond knowledge is the highest
state, the two extremes of which seem the same. Knowledge itself is a
manufactured something, a combination; it is not reality. \\

What is the result of constant practice of this higher
concentration? All old tendencies of restlessness and dullness will be
destroyed, as well as the tendencies of goodness too. The case is
similar to that of the chemicals used to take the dirt and alloy off
gold. When the ore is smelted down, the dross is burnt along with the
chemicals. So this constant controlling power will stop the previous
bad tendencies, and eventually, the good ones also. Those good and evil
tendencies will suppress each other, leaving alone the Soul, in its own
splendour untrammelled by either good or bad, the omnipresent,
omnipotent, and omniscient. Then the man will know that he had neither
birth nor death, nor need for
heaven or earth. He will know that he neither came nor went, it was
nature which was moving, and that movement was reflected upon the soul.
The form of the light reflected by the glass upon the wall moves, and
the wall foolishly thinks it is moving. So with all of us; it is the
Chitta constantly moving making itself into various forms, and we think
that we are these various forms. All these delusions will vanish. When
that free Soul will command — not pray or beg, but command — then
whatever It desires will be immediately fulfilled; whatever It wants It
will be able to do. According to the Sankhya philosophy, there is no
God. It says that there can be no God of this universe, because if
there were one, He must be a soul, and a soul must be either bound or
free. How can the soul that is bound by nature, or controlled by
nature, create? It is itself a slave. On the other hand, why should the
Soul that is free create and manipulate all these things? It has no
desires, so it cannot have any need to create. Secondly, it says the
theory of God is an unnecessary one; nature explains all. What is the
use of any God? But Kapila teaches that there are many souls, who,
though nearly attaining perfection, fall short because they cannot
perfectly renounce all powers. Their minds for a time merge in nature,
to re-emerge as its masters. Such gods there are. We shall all become
such gods, and, according to the Sankhyas, the God spoken of in the
Vedas really means one of these free souls. Beyond them there is not an
eternally free and blessed Creator of the universe. On the other hand,
the Yogis say, "Not so, there is a God; there is one Soul separate from
all other souls, and He is the eternal Master of all creation, the ever
free, the Teacher of all teachers." The Yogis admit that those whom the
Sankhyas call "the merged in nature" also exist. They are Yogis who
have fallen short of perfection, and though, for a time, debarred from
attaining the goal, remain as rulers of parts of the universe. \\

\begin{center}
\begin{sanskrit}
भव-प्रत्ययो विदेह-प्रकृतिलयानाम् ॥१९॥
\end{sanskrit}
\end{center}
19. (This Samadhi when not followed by extreme non-attachment)
becomes the cause of the re-manifestation of the gods and of those that
become merged in nature. \\

The gods in the Indian systems of philosophy represent certain
high offices which are filled successively by various souls. But none
of them is perfect. \\

\begin{center}
\begin{sanskrit}
श्रद्धा-वीर्य-स्मृति-समाधि-प्रज्ञा-पूर्वक
इतरेषाम् ॥२०॥
\end{sanskrit}
\end{center}
20. To others (this Samadhi) comes through faith, energy,
memory, concentration, and discrimination of the real. \\

These are they who do not want the position of gods or even
that of rulers of cycles. They attain to liberation.\\

\begin{center}
\begin{sanskrit}
तीव्रसंवेगानामासन्नः ॥२१॥
\end{sanskrit}
\end{center}
21. Success is speedy for the extremely energetic. \\

\begin{center}
\begin{sanskrit}
मृदुमध्याधिमात्रत्वात्ततोऽपि विशेषः ॥२२॥
\end{sanskrit}
\end{center}
22. The success of Yogis differs according as the means they
adopt are mild, medium, or intense. \\

\begin{center}
\begin{sanskrit}
ईश्वरप्रणिधानाद्वा ॥२३॥
\end{sanskrit}
\end{center}
23. Or by devotion to Ishvara. \\

\begin{center}
\begin{sanskrit}
क्लेशकर्मविपाकाशयैरपरामृष्टः पुरुषविशेष
ईश्वरः ॥२४॥
\end{sanskrit}
\end{center}
24. Ishvara (the Supreme Ruler) is a special Purusha,
untouched by misery, actions, their results, and desires. \\

We must again remember that the Pâtanjala Yoga philosophy is
based upon the Sankhya philosophy; only in the latter there is no place
for God, while with the Yogis God has a place. The Yogis, however, do
not mention many ideas about
God, such as creating. God as the Creator of the universe is not meant
by the Ishvara of the Yogis. According to the Vedas, Ishvara is the
Creator of the universe; because it is harmonious, it must be the
manifestation of one will. The Yogis want to establish a God, but they
arrive at Him in a peculiar fashion of their own. They say: \\

\begin{center}
\begin{sanskrit}
तत्र निरतिशयं सर्वज्ञत्वबीजम् ॥२५॥
\end{sanskrit}
\end{center}
25. In Him becomes infinite that all-knowingness which in
others is (only) a germ. \\

The mind must always travel between two extremes. You can
think of limited space, but that very idea gives you also unlimited
space. Close your eyes and think of a little space; at the same time
that you perceive the little circle, you have a circle round it of
unlimited dimensions. It is the same with time. Try to think of a
second; you will have, with the same act of perception, to think of
time which is unlimited. So with knowledge. Knowledge is only a germ in
man, but you will have to think of infinite knowledge around it, so
that the very constitution of our mind shows us that there is unlimited
knowledge, and the Yogis call that unlimited knowledge God. \\

\begin{center}
\begin{sanskrit}
स पूर्वेषामपि गुरुः कालेनानवच्छेदात् ॥२६॥
\end{sanskrit}
\end{center}
26. He is the Teacher of even the ancient teachers, being not
limited by time. \\

It is true that all knowledge is within ourselves, but this
has to be called forth by another knowledge. Although the capacity to
know is inside us, it must be called out, and that calling out of
knowledge can only be done, a Yogi maintains, through another
knowledge. Dead, insentient matter never calls out knowledge, it is the
action of knowledge that brings out knowledge. Knowing beings must be
with us to call forth what is in
us, so these teachers were always necessary. The world was never
without them, and no knowledge can come without them. God is the
Teacher of all teachers, because these teachers, however great they may
have been — gods or angels — were all bound and limited by time, while
God is not. There are two peculiar deductions of the Yogis. The first
is that in thinking of the limited, the mind must think of the
unlimited; and that if one part of that perception is true, so also
must the other be, for the reason that their value as perceptions of
the mind is equal. The very fact that man has a little knowledge shows
that God has unlimited knowledge. If I am to take one, why not the
other? Reason forces me to take both or reject both. If I believe that
there is a man with a little knowledge, I must also admit that there is
someone behind him with unlimited knowledge. The second deduction is
that no knowledge can come without a teacher. It is true, as the modern
philosophers say, that there is something in man which evolves out of
him; all knowledge is in man, but certain environments are necessary to
call it out. We cannot find any knowledge without teachers. If there
are men teachers, god teachers, or angel teachers, they are all
limited; who was the teacher before them? We are forced to admit, as a
last conclusion, one teacher who is not limited by time; and that One
Teacher of infinite knowledge, without beginning or end, is called God.
\\

\begin{center}
\begin{sanskrit}
तस्य वाचकः प्रणवः ॥२७॥
\end{sanskrit}
\end{center}
27. His manifesting word is Om. \\

None\\

\begin{center}
\begin{sanskrit}
तज्जपस्तदर्थभावनम् ॥२८॥
\end{sanskrit}
\end{center}
28. The repetition of this (Om) and meditating on its meaning
(is the way). \\

Why should there be repetition? We have not forgotten the
theory of Samskaras, that the sum-total of
impressions lives in the mind. They become more and more latent but
remain there, and as soon as they get the right stimulus, they come
out. Molecular vibration never ceases. When this universe is destroyed,
all the massive vibrations disappear; the sun, moon, stars, and earth,
melt down; but the vibrations remain in the atoms. Each atom performs
the same function as the big worlds do. So even when the vibrations of
the Chitta subside, its molecular vibrations go on, and when they get
the impulse, come out again. We can now understand what is meant by
repetition. It is the greatest stimulus that can be given to the
spiritual Samskaras. "One moment of company with the holy makes a ship
to cross this ocean of life." Such is the power of association. So this
repetition of Om, and thinking of its meaning, is keeping good company
in your own mind. Study, and then meditate on what you have studied.
Thus light will come to you, the Self will become manifest. \\

But one must think of Om, and of its meaning too. Avoid evil
company, because the scars of old wounds are in you, and evil company
is just the thing that is necessary to call them out. In the same way
we are told that good company will call out the good impressions that
are in us, but which have become latent. There is nothing holier in the
world than to keep good company, because the good impressions will then
tend to come to the surface. \\

\begin{center}
\begin{sanskrit}
ततः प्रत्यक्चेतनाधिगमोऽप्यन्तरायाभावश्च ॥२९॥
\end{sanskrit}
\end{center}
29. From that is gained (the knowledge of) introspection, and
the destruction of obstacles. \\

The first manifestation of the repetition and thinking of Om
is that the introspective power will manifest more and more, all the
mental and physical obstacles will begin to vanish. What are the
obstacles to the Yogi? \\

\begin{center}
\begin{sanskrit}
व्याधि-स्त्यान-संशय-प्रमादालस्याविरति-भ्रान्तिदर्शनालब्धभूमिकत्वानवस्थितत्वानि
चित्तविक्षेपास्तेऽन्तरायाः ॥३०॥
\end{sanskrit}
\end{center}
30. Disease, mental laziness, doubt, lack of enthusiasm,
lethargy, clinging to sense-enjoyments, false perception, non-attaining
concentration, and falling away from the state when obtained, are the
obstructing distractions.\\

None\\

\begin{center}
\begin{sanskrit}
दुःख-दौर्मनस्याङ्गमेजयत्व-श्वासप्रश्वासा
विक्षेपसहभुवः ॥३१॥
\end{sanskrit}
\end{center}
31. Grief, mental distress, tremor of the body, irregular
breathing, accompany non-retention of concentration. \\

Concentration will bring perfect repose to mind and body every
time it is practised. When the practice has been misdirected, or not
enough controlled, these disturbances come. Repetition of Om and
self-surrender to the Lord will strengthen the mind, and bring fresh
energy. The nervous shakings will come to almost everyone. Do
not mind them at all, but keep on practising. Practice will cure them
and make the seat firm. \\

\begin{center}
\begin{sanskrit}
तत्प्रतिषेधार्थमेकतत्त्वाभ्यासः ॥३२॥
\end{sanskrit}
\end{center}
32. To remedy this, the practice of one subject (should be
made). \\

Making the mind take the form of one object for some time will
destroy these obstacles. This is general advice. In the following
aphorisms it will be expanded and particularized. As one practice
cannot suit everyone, various methods will be advanced, and everyone by
actual experience will find out that which helps him most. \\

\begin{center}
\begin{sanskrit}
None
\end{sanskrit}
\end{center}
33. Friendship, mercy, gladness, and indifference, being
thought of in regard to subjects, happy, unhappy, good, and evil
respectively, pacify the Chitta. \\

We must have these four sorts of ideas. We must have
friendship for all; we must be merciful towards those that are in
misery; when people are happy, we ought to be happy; and to the wicked
we must be indifferent. So with all subjects that come before us. If
the subject is a good one, we shall feel friendly towards it; if the
subject of thought is one that is miserable, we must be merciful
towards it. If it is good, we must be glad; if it is evil, we must be
indifferent. These attitudes of the mind towards the different subjects
that come before it will make the mind peaceful. Most of our
difficulties in our daily lives come from being unable to hold our
minds in this way. For instance, if a man does evil to us, instantly we
want to react evil, and every reaction of evil shows that we are not
able to hold the Chitta down; it comes out in waves towards the object,
and we lose our power. Every reaction in the form of hatred or evil is
so much loss to the mind; and
every evil thought or deed of hatred, or any thought of reaction, if it
is controlled, will be laid in our favour. It is not that we lose by
thus restraining ourselves; we are gaining infinitely more than we
suspect. Each time we suppress hatred, or a feeling of anger, it is so
much good energy stored up in our favour; that piece of energy will be
converted into the higher powers.\\

\begin{center}
\begin{sanskrit}
प्रच्छर्दन-विधारणाभ्यां वा प्राणस्य ॥३४॥
\end{sanskrit}
\end{center}
34. By throwing out and restraining the Breath. \\

The word used is Prâna. Prana is not exactly breath. It is the
name for the energy that is in the universe. Whatever you see in the
universe, whatever moves or works, or has life, is a manifestation of
this Prana. The sum-total of the energy displayed in the universe is
called Prana. This Prana, before a cycle begins, remains in an almost
motionless state; and when the cycle begins, this Prana begins to
manifest itself. It is this Prana that is manifested as motion — as the
nervous motion in human beings or animals; and the same Prana is
manifesting as thought, and so on. The whole universe is a combination
of Prana and Âkâsha; so is the human body. Out of Akasha you get the
different materials that you feel and see, and out of Prana all the
various forces. Now this throwing out and restraining the Prana is what
is called Pranayama. Patanjali, the father of the Yoga philosophy, does
not give very many particular directions about Pranayama, but later on
other Yogis found out various things about this Pranayama, and made of
it a great science. With Patanjali it is one of the many ways, but he
does not lay much stress on it. He means that you simply throw the air
out, and draw it in, and hold it for some time, that is all, and by
that, the mind will become a little calmer. But, later on, you will
find that out of this is evolved a particular science called Pranayama.
We shall hear a little of what these later Yogis have to say. \\

Some of this I have told you before, but a little repetition
will serve to fix it in your minds. First, you must remember that this
Prana is not the breath; but that which causes the motion of the
breath, that which is the vitality of the breath, is the Prana. Again,
the word Prana is used for all the senses; they are all called Pranas,
the mind is called Prana; and so we see that Prana is force. And yet we
cannot call it force, because force is only the manifestation of it. It
is that which manifests itself as force and everything else in the way
of motion. The Chitta, the mind-stuff, is the engine which draws in the
Prana from the surroundings, and manufactures out of Prana the various
vital forces — those that keep the body in preservation — and thought,
will, and all other powers. By the abovementioned process of breathing
we can control all the various motions in the body, and the various
nerve currents that are running through the body. First we begin to
recognise them, and then we slowly get control over them. \\

Now, these later Yogis consider that there are three main
currents of this Prana in the human body. One they call Idâ, another
Pingalâ, and the third Sushumnâ. Pingala, according to them, is on the
right side of the spinal column, and the Ida on the left, and in the
middle of the spinal column is the Sushumna, an empty channel. Ida and
Pingala, according to them, are the currents working in every man, and
through these currents, we are performing all the functions of life.
Sushumna is present in all, as a possibility; but it works only in the
Yogi. You must remember that Yoga changes the body. As you go on
practising, your body changes; it is not the same body that you had
before the practice. That is very rational, and can be explained,
because every new thought that we have must make, as it were, a new
channel through the brain, and that explains the tremendous
conservatism of human nature. Human nature likes to run through the
ruts that are already there, because it is easy. If we think, just for
example's sake, that the mind is like a needle, and the brain substance
a soft lump before it, then each thought that we have makes a street,
as it were, in the brain, and this street would close up, but for the
grey matter which comes and makes a lining to keep it separate. If
there were no grey matter, there would be no memory, because memory
means going over these old streets, retracing a thought as it were. Now
perhaps you have marked that when one talks on subjects in which one
takes a few ideas that are familiar to everyone, and combines and
recombines them, it is easy to follow because these channels are
present in everyone's brain, and it is only necessary to recur to them.
But whenever a new subject comes, new channels have to be made, so it
is not understood readily. And that is why the brain (it is the brain,
and not the people themselves) refuses unconsciously to be acted upon
by new ideas. It resists. The Prana is trying to make new channels, and
the brain will not allow it. This is the secret of conservatism. The
fewer channels there have been in the brain, and the less the needle of
the Prana has made these passages, the more conservative will be the
brain, the more it will struggle against new thoughts. The more
thoughtful the man, the more complicated will be the streets in his
brain, and the more easily he will take to new ideas, and understand
them. So with every fresh idea, we make a new impression in the brain,
cut new channels through the brain-stuff, and that is why we find that
in the practice of Yoga (it being an entirely new set of thoughts and
motives) there is so much physical resistance at first. That is why we
find that the part of religion which deals with the world-side of
nature is so widely accepted, while the other part, the philosophy, or
the psychology, which clears with the inner nature of man, is so
frequently neglected. \\

We must remember the definition of this world of ours; it is
only the Infinite Existence projected into the
plane of consciousness. A little of the Infinite is projected into
consciousness, and that we call our world. So there is an Infinite
beyond; and religion has to deal with both — with the little lump we
call our world, and with the Infinite beyond. Any religion which deals
with one only of these two will be defective. It must deal with both.
The part of religion which deals with the part of the Infinite which
has come into the plane of consciousness, got itself caught, as it
were, in the plane of consciousness, in the cage of time, space, and
causation, is quite familiar to us, because we are in that already, and
ideas about this world have been with us almost from time immemorial.
The part of religion which deals with the Infinite beyond comes
entirely new to us, and getting ideas about it produces new channels in
the brain, disturbing the whole system, and that is why you find in the
practice of Yoga ordinary people are at first turned out of their
grooves. In order to lessen these disturbances as much as possible, all
these methods are devised by Patanjali, that we may practice any one of
them best suited to us. \\

\begin{center}
\begin{sanskrit}
विषयवती वा प्रवृत्तिरुत्पन्ना मनसः
स्थितिनिबन्धिनी ॥३५॥
\end{sanskrit}
\end{center}
35. Those forms of concentration that bring extraordinary
sense-perceptions cause perseverance of the mind. \\

This naturally comes with Dhâranâ, concentration; the Yogis
say, if the mind becomes concentrated on the tip of the nose, one
begins to smell, after a few days, wonderful perfumes. If it becomes
concentrated at the root of the tongue, one begins to hear sounds; if
on the tip of the tongue, one begins to taste wonderful flavours; if on
the middle of the tongue, one feels as if one were coming in contact
with something. If one concentrates one's mind on the palate, one
begins to see peculiar things. If a man whose mind is disturbed wants
to take up some of these practices of Yoga, yet doubts the truth of
them, he will have his doubts
set at rest when, after a little practice, these things come to him,
and he will persevere. \\

\begin{center}
\begin{sanskrit}
None
\end{sanskrit}
\end{center}
36. Or (by the meditation on) the Effulgent Light, which is
beyond all sorrow. \\

This is another sort of concentration. Think of the lotus of
the heart, with petals downwards, and running through it, the Sushumna;
take in the breath, and while throwing the breath out imagine that the
lotus is turned with the petals upwards, and inside that lotus is an
effulgent light. Meditate on that. \\

\begin{center}
\begin{sanskrit}
वीतरागविषयं वा चित्तम् ॥३७॥
\end{sanskrit}
\end{center}
37. Or (by meditation on) the heart that has given up all
attachment to sense-objects. \\

Take some holy person, some great person whom you revere, some
saint whom you know to be perfectly nonattached, and think of his
heart. That heart has become non-attached, and meditate on that heart;
it will calm the mind. If you cannot do that, there is the next way:\\

\begin{center}
\begin{sanskrit}
स्वप्ननिद्राज्ञानालम्बनं वा ॥३८॥
\end{sanskrit}
\end{center}
38. Or by meditating on the knowledge that comes in sleep. \\

Sometimes a man dreams that he has seen angels coming to him
and talking to him, that he is in an ecstatic condition, that he has
heard music floating through the air. He is in a blissful condition in
that dream, and when he wakes, it makes a deep impression on him. Think
of that dream as real, and meditate upon it. If you cannot do that,
meditate on any holy thing that pleases you. \\

\begin{center}
\begin{sanskrit}
यथाभिमतध्यानाद्वा ॥३९॥
\end{sanskrit}
\end{center}
39. Or by the meditation on anything that appeals to one as
good. \\

This does not mean any wicked subject, but anything good that
you like, any place that you like best, any scenery that you like best,
any idea that you like best, anything that will concentrate the mind. \\

\begin{center}
\begin{sanskrit}
परमाणु परममहत्त्वान्तोऽस्य वशीकारः ॥४०॥
\end{sanskrit}
\end{center}
40. The Yogi's mind thus meditating, becomes unobstructed from
the atomic to the infinite. \\

The mind, by this practice, easily contemplates the most
minute, as well as the biggest thing. Thus the mindwaves become
fainter. \\

\begin{center}
\begin{sanskrit}
क्षीणवृत्तेरभिजातस्येव
मणेर्ग्रहीतृ-ग्रहण-ग्राह्येषु
तत्स्थ-तदञ्जनता समापत्तिः ॥४१॥
\end{sanskrit}
\end{center}
41. The Yogi whose Vrittis have thus become powerless
(controlled) obtains in the receiver, (the instrument of) receiving,
and the received (the Self, the mind, and external objects),
concentratedness arid sameness like the crystal (before different
coloured objects). \\

What results from this constant meditation? We must remember
how in a previous aphorism Patanjali went into the various states of
meditation, how the first would be the gross, the second the fine, and
from them the advance was to still finer objects. The result of these
meditations is that we can meditate as easily on the fine as on the
gross objects. Here the Yogi sees the three things, the receiver, the
received, and the receiving instrument, corresponding to the Soul,
external objects, and the mind. There are three objects of meditation
given us. First, the gross things, as bodies, or material objects;
second, fine things, as the mind, the Chitta; and third, the Purusha
qualified, not the Purusha itself, but the Egoism. By practice, the
Yogi gets established in all these meditations. Whenever he meditates
he can keep out all other
thoughts; he becomes identified with that on which he meditates. When
he meditates, he is like a piece of crystal. Before flowers the crystal
becomes almost identified with the flowers. If the flower is red, the
crystal looks red, or if the flower is blue, the crystal looks blue. \\

\begin{center}
\begin{sanskrit}
तत्र शब्दार्थज्ञानविकल्पैः सङ्कीर्णा
सवितर्का समापत्तिः ॥४२॥
\end{sanskrit}
\end{center}
42. Sound, meaning, and resulting knowledge, being mixed up,
is (called) Samadhi with question. \\

Sound here means vibration, meaning the nerve currents which
conduct it; and knowledge, reaction. All the various meditations we
have had so far, Patanjali calls Savitarka (meditation with question).
Later on he gives us higher and higher Dhyânas. In these that are
called "with question," we keep the duality of subject and object,
which results from the mixture of word, meaning, and knowledge. There
is first the external vibration, the word. This, carried inward by the
sense currents, is the meaning. After that there comes a reactionary
wave in the Chitta, which is knowledge, but the mixture of these three
makes up what we call knowledge. In all the meditations up to this we
get this mixture as objects of meditation. The next Samadhi is higher. \\

\begin{center}
\begin{sanskrit}
None
\end{sanskrit}
\end{center}
स्मृतिपरिशुद्धौ स्वरूपशून्येवार्थमात्रनिर्भासा निर्वितर्का ॥४३॥\\

43. The Samadhi called "without question" (comes) when the
memory is purified, or devoid of qualities, expressing only the meaning
(of the meditated object). \\

It is by the practice of meditation of these three that we
come to the state where these three do not mix. We can get rid of them.
We will first try to understand what these three are. Here is the
Chitta; you will always remember
the simile of the mind-stuff to a lake, and the vibration, the word,
the sound, like a pulsation coming over it. You have that calm lake in
you, and I pronounce a word, "Cow". As soon as it enters through your
ears there is a wave produced in your Chitta along with it. So that
wave represents the idea of the cow, the form or the meaning as we call
it. The apparent cow that you know is really the wave in the mind-stuff
that comes as a reaction to the internal and external sound vibrations.
With the sound, the wave dies away; it can never exist without a word.
You may ask how it is, when we only think of the cow, and do not hear a
sound. You make that sound yourself. You are saying "cow" faintly in
your mind, and with that comes a wave. There cannot be any wave without
this impulse of sound; and when it is not from outside, it is from
inside, and when the sound dies, the wave dies. What remains? The
result of the reaction, and that is knowledge. These three are so
closely combined in our mind that we cannot separate them. When the
sound comes, the senses vibrate, and the wave rises in reaction; they
follow so closely upon one another that there is no discerning one from
the other. When this meditation has been practiced for a long time,
memory, the receptacle of all impressions, becomes purified, and we are
able clearly to distinguish them from one another. This is called
Nirvitarka, concentration without question. \\

\begin{center}
\begin{sanskrit}
एतयैव सविचारा निर्विचारा च सूक्ष्मविषया
व्याख्याता ॥४४॥
\end{sanskrit}
\end{center}
44. By this process (the concentrations) with discrimination
and without discrimination, whose objects are finer, are (also)
explained. \\

A process similar to the preceding is applied again; only, the
objects to be taken up in the former meditations are gross; in this
they are fine. \\

\begin{center}
\begin{sanskrit}
सूक्ष्मविषयत्वञ्चालिङ्ग-पर्यवसानम् ॥४५॥
\end{sanskrit}
\end{center}
45. The finer objects end with the Pradhâna. \\

The gross objects are only the elements and everything
manufactured out of them. The fine objects begin with the Tanmatras or
fine particles. The organs, the mind, (The mind, or common sensorium,
the aggregate of all the senses), egoism, the mind-stuff (the cause of
all manifestation), the equilibrium state of Sattva, Rajas, and Tamas
materials — called Pradhâna (chief), Prakriti (nature), or Avyakta
(unmanifest) — are all included within the category of fine objects,
the Purusha (the Soul) alone being excepted. \\

\begin{center}
\begin{sanskrit}
ता एव सबीजः समाधिः ॥४६॥
\end{sanskrit}
\end{center}
46. These concentrations are with seed. \\

These do not destroy the seeds of past actions, and thus
cannot give liberation, but what they bring to the Yogi is stated in
the following aphorism. \\

\begin{center}
\begin{sanskrit}
निर्विचार-वैशारद्येऽध्यात्मप्रसादः ॥४७॥
\end{sanskrit}
\end{center}
47. The concentration "without discrimination" being purified,
the Chitta becomes firmly fixed. \\

\begin{center}
\begin{sanskrit}
ऋतम्भरा तत्र प्रज्ञा ॥४८॥
\end{sanskrit}
\end{center}
48. The knowledge in that is called "filled with Truth". \\

The next aphorism will explain this. \\

\begin{center}
\begin{sanskrit}
श्रुतानुमानप्रज्ञाभ्यामन्यविषया
विशेषार्थत्वात् ॥४९॥
\end{sanskrit}
\end{center}
49. The knowledge that is gained from testimony and inference
is about common objects. That from the Samadhi just mentioned is of a
much higher order, being able to penetrate where inference and
testimony cannot go. \\

The idea is that we have to get our knowledge of ordinary
objects by direct perception, and by inference therefrom, and from
testimony of people who are competent. By "people who are competent,"
the Yogis always mean the Rishis, or the Seers of the thoughts recorded
in the scriptures — the Vedas. According to them, the only proof of the
scriptures is that they were the testimony of competent persons, yet
they say the scriptures cannot take us to realisation. We can read all
the Vedas, and yet will not realise anything, but when we practise
their teachings, then we attain to that state which realises what the
scriptures say, which penetrates where neither reason nor perception
nor inference can go, and where the testimony of others cannot avail.
This is what is meant by the aphorism. \\

Realisation is real religion, all the rest is only preparation
— hearing lectures, or reading books, or reasoning is merely preparing
the ground; it is not religion. Intellectual assent and intellectual
dissent are not religion. The central idea of the Yogis is that just as
we come in direct contact with objects of the senses, so religion even
can be directly perceived in a far more intense sense. The truths of
religion, as God and Soul, cannot be perceived by the external senses.
I cannot see God with my eyes, nor can I touch Him with my hands, and
we also know that neither can we reason beyond the senses. Reason
leaves us at a point quite indecisive; we may reason all our lives, as
the world has been doing for thousands of years, and the result is that
we find we are incompetent to prove or disprove the facts of religion.
What we perceive directly we take as the basis, and upon that basis we
reason. So it is obvious that reasoning has to run within these bounds
of perception. It can never go beyond. The whole scope of realisation,
therefore, is beyond sense-perception. The Yogis say that man can go
beyond his direct sense-perception, and beyond his reason also. Man has
in him the faculty, the power,
of transcending his intellect even, a power which is in every being,
every creature. By the practice of Yoga that power is aroused, and then
man transcends the ordinary limits of reason, and directly perceives
things which are beyond all reason. \\

\begin{center}
\begin{sanskrit}
तज्जः संस्कारोऽन्यसंस्कारप्रतिबन्धी ॥५०॥
\end{sanskrit}
\end{center}
50. The resulting impression from this Samadhi obstructs all
other impressions. \\

We have seen in the foregoing aphorism that the only way of
attaining to that superconsciousness is by concentration, and we have
also seen that what hinder the mind from concentration are the past
Samskaras, impressions. All of you have observed that, when you are
trying to concentrate your mind, your thoughts wander. When you are
trying to think of God, that is the very time these Samskaras appear.
At other times they are not so active; but when you want them not, they
are sure to be there, trying their best to crowd in your mind. Why
should that be so? Why should they be much more potent at the time of
concentration? It is because you are repressing them, and they react
with all their force. At other times they do not react. How countless
these old past impressions must be, all lodged somewhere in the Chitta,
ready, waiting like tigers, to jump up! These have to be suppressed
that the one idea which we want may arise, to the exclusion of the
others. Instead they are all struggling to come up at the same time.
These are the various powers of the Samskaras in hindering
concentration of the mind. So this Samadhi which has just been given is
the best to be practised, on account of its power of suppressing the
Samskaras. The Samskara which will be raised by this sort of
concentration will be so powerful that it will hinder the action of the
others, and hold them in check. \\

\begin{center}
\begin{sanskrit}
तस्यापि निरोधे सर्वनिरोधान्निर्बीजः समाधिः
॥५१॥
\end{sanskrit}
\end{center}
51. By the restraint of even this (impression, which obstructs
all other impressions), all being restrained, comes the "seedless"
Samadhi. \\

You remember that our goal is to perceive the Soul itself. We
cannot perceive the Soul, because it has got mingled up with nature,
with the mind, with the body. The ignorant man thinks his body is the
Soul. The learned man thinks his mind is the Soul. But both of them are
mistaken. What makes the Soul get mingled up with all this? Different
waves in the Chitta rise and cover the Soul; we only see a little
reflection of the Soul through these waves; so, if the wave is one of
anger, we see the Soul as angry; "I am angry," one says. If it is one
of love, we see ourselves reflected in that wave, and say we are
loving. If that wave is one of weakness, and the Soul is reflected in
it, we think we are weak. These various ideas come from these
impressions, these Samskaras covering the Soul. The real nature of the
Soul is not perceived as long as there is one single wave in the lake
of the Chitta; this real nature will never be perceived until all the
waves have subsided. So, first, Patanjali teaches us the meaning of
these waves; secondly, the best way to repress them; and thirdly, how
to make one wave so strong as to suppress all other waves, fire eating
fire as it were. When only one remains, it will be easy to suppress
that also, and when that is gone, this Samadhi or concentration is
called seedless. It leaves nothing, and the Soul is manifested just as
It is, in Its own glory. Then alone we know that the Soul is not a
compound; It is the only eternal simple in the universe, and as such,
It cannot be born, It cannot die; It is immortal, indestructible, the
ever-living essence of intelligence. \\

